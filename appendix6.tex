%-------------------------------------------------------------------------------------
%-------------------Setup Matlab Listings syntax---------------------------
%-------------------------------------------------------------------------------------
\definecolor{mygreen}{RGB}{28,172,0} % color values Red, Green, Blue
\definecolor{mylilas}{RGB}{170,55,241}
\lstset{language=Matlab,%
    %basicstyle=\color{red},
    breaklines=true,%
    morekeywords={matlab2tikz},
    keywordstyle=\color{blue},%
    morekeywords=[2]{1}, keywordstyle=[2]{\color{black}},
    identifierstyle=\color{black},%
    stringstyle=\color{mylilas},
    commentstyle=\color{mygreen},%
    showstringspaces=false,%without this there will be a symbol in the places where there is a space
    numbers=left,%
    numberstyle={\tiny \color{black}},% size of the numbers
    numbersep=9pt, % this defines how far the numbers are from the text
    emph=[1]{for,end,break},emphstyle=[1]\color{red}, %some words to emphasise
    %emph=[2]{word1,word2}, emphstyle=[2]{style},    
}
%-------------------------------------------------------------------------------------
%-------------------------------------------------------------------------------------
\section{Euler vs Trapezoidal Method}
%-------------------------------------------------------------------------------------
\begin{lstlisting}[language=matlab]
clear, format long, clc, close all
dt = 1;

t0 = [0:.01:4];

y0 = exp(t0);

%Euler integration

t1 = [0:4];
y1 = exp(0);
for i=1:4
    y1(i+1)=y1(i) + dt*exp(i-1);
end

%Trapezoidal integration
t2 = [0:4];
y2 = exp(0);

for i=1:4
    y2(i+1)=y2(i) + dt*exp(i-1);
    y2(i+1)=y2(i) + (dt/2)*(exp(i-1)+y2(i+1));
end

plot(t0,y0,'k--',t1,y1,'b',t2,y2,'r')
legend('y=e^t','Euler Method','Trapezoidal Method')
\end{lstlisting}

\section{SISO Lyapunov Solution Proof}
%-------------------------------------------------------------------------------------
\begin{lstlisting}[language=matlab]
clear,clc, format compact, close all
%% Find Pb
wn = [5:0.1:10]; %rad/s
for i=1:length(wn)
    b = [wn(i)];
    a = [1,wn(i)];
    [A,B,C,D] = tf2ss(b,a)

    Pb(i) = lyap(A,1);
end
Pb_test = 1./(2*wn);

plot(wn,Pb_test,'o',wn,Pb)

%in this simple 1x1 matrix case Pb is easy to calc by hand in code
% Pb ends up being: Pb=1/2*wn;
\end{lstlisting}

\section{Reverse Linear Chirp}
%-------------------------------------------------------------------------------------
\begin{lstlisting}[language=matlab]
clear, format compact, clc, close all

fs = 1/200;
t = [0:fs:7];
f0= 0.01;%Hz
f1= 10; %Hz
k = (f1-f0)/(t(end));

phi_0 = 0.0;

sample_delay = 0.02;

out = sin(phi_0+2*pi*(f0*(7-(t+sample_delay))+(k/2).*(7-(t+sample_delay)).^2));
%out = 1500 + out*10;

out = out*10;

plot(t,out)
\end{lstlisting}

\section{Projection Operator Example Plots}\label{sec:projection_proof}
%-------------------------------------------------------------------------------------
\begin{lstlisting}[language=matlab]
clear, clc, format compact, close all

%% Plot Projection
epsilon = [0.25,0.28,0.3,0.38];
theta_max = 0.65;
theta_min = 0.25;
center = (theta_max+theta_min)/2;

span = (theta_max-theta_min)*1.4;
theta = [center-(span/2):0.01:center+(span/2)];

for j=1:length(epsilon)
    for i=1:length(theta)          
        %f_theta(i,j) = -(theta(i).^2-theta_max^2)/(epsilon(j)*theta_max^2);
        %f_theta_dot(i,j) = -(2*theta(i))/(epsilon(j)*theta_max^2);
        f_theta(i,j) = -(theta_min-theta(i))*(theta_max-theta(i))./(epsilon(j));
        f_theta_dot(i,j) = (theta_min+theta_max-(2*theta(i)))/(epsilon(j));
    end
end 

figure
for j=1:length(epsilon)   
hold on
plot(theta,f_theta(:,j))
end
line([min(theta),max(theta)],[0,0],'color','blue','linestyle','--')
line([theta_min,theta_min],[min(f_theta(:,1)),max(f_theta(:,1))],'color','red','linestyle','--')
line([theta_max,theta_max],[min(f_theta(:,1)),max(f_theta(:,1))],'color','red','linestyle','--')
legend('\epsilon=0.25','\epsilon=0.28','\epsilon=0.30','\epsilon=0.38')
ylabel('f(\theta)')
xlabel('\theta')
title('Projection Operator')
hold off
\end{lstlisting}

\section{System Identification}\label{sec:sysid_matlab}
%-------------------------------------------------------------------------------------
\begin{lstlisting}[language=matlab]
clear, clc, format compact, close all

% Import data file for analysis
[filename, pathname] = uigetfile('*.m','Choose first MATLAB file');
run(filename)
%%
% Clean tigger PWMs
for r=1:length(RCIN.data(:,9))    
    if RCIN.data(r,8) < 1700 %channel 6
        RCIN.data(r,8) = 1000;
    end
    if RCIN.data(r,9) < 1700 %channel 7
        RCIN.data(r,9) = 1000;
    end
end

%Trigger Channel Data
roll_trigger_data = RCIN.data(:,9);
pitch_trigger_data = RCIN.data(:,8);
trigger_data_time = RCIN.data(:,2);
%Command Data
roll_command_data = ((RCOU.data(:,3)-1500)+(RCOU.data(:,4)-1500))+1526;
pitch_command_data = ((RCOU.data(:,3)-1500)-(RCOU.data(:,4)-1500))+1460;
command_data_time = RCOU.data(:,2);
%IMU Data
p = IMU2.data(:,3);
q = -IMU2.data(:,4);
IMU_time = IMU2.data(:,2);

% Plot to show sectioned data
plot(RCIN.data(:,2)*10^-6,RCIN.data(:,8)-1500,IMU2.data(:,2)*10^-6,p*(180/pi),...
     RCIN.data(:,2)*10^-6,RCIN.data(:,9)-1500,IMU2.data(:,2)*10^-6,q*(180/pi))
legend('ch6','p','ch7','q')
xlabel('time (seconds)')
ylabel('pwm counts')

[start_index, stop_index] = find_trigger(roll_trigger_data, 1700);
fprintf('Which test would you like to analyze? [Pick a number between 1 and %i]\n',length(start_index));
num_of_test_desired = input('');

%[ input, output, time ] = section_triggered_data( input_data, input_data_time, output_data, output_data_time, trigger_data, trigger_time, num_of_test_desired )
[ input, output, time ] = section_triggered_data( roll_command_data, command_data_time, p, IMU_time, roll_trigger_data, trigger_data_time, num_of_test_desired );

%Center controls for SysID and plotting
input = input-1500; %center stick ouputs 1500

% Theoretical Chirp
fs = 1/50;
t = [0:fs:7];
f0= 0.01;%Hz
f1= 10; %Hz
k = (f1-f0)/(t(end));
phi_0 = 0;

input_theoretical = sin(phi_0+2*pi*(f0*(7-t)+(k/2).*(7-t).^2));
input_theoretical = input_theoretical*75;

figure
plot(time,input,'o', time, output*(180/pi),t,input_theoretical)
legend('roll cmd', 'roll rate')

%% System Identification
loop_rate = mean(diff(IMU_time))*10^-6;
time_clean = [0:loop_rate:(length(time)-1)*loop_rate]';
output_clean = interp1(time,output,time_clean,'pchip');
%input_clean = interp1(time,input,time_clean,'pchip');
input_clean = interp1(t,input_theoretical,time_clean,'pchip');


figure
plot(time, input_clean, time, output_clean*(180/pi))
legend('roll cmd', 'roll rate')

[y,t,x] = estimate_model(input_clean, output_clean, time_clean);

figure
subplot(2,1,1)
plot(time_clean,input_clean)
legend('command')
xlabel('time (seconds)')
ylabel('\delta (pwm)')
title('Roll Analysis')
subplot(2,1,2)
plot(t,y,time_clean,output_clean)
legend('model','measured response')
xlabel('Time (s)')
ylabel('Roll rate (rad/s)')
\end{lstlisting}
