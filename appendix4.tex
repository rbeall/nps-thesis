\label{appendix:lyapunov}

Aerospace designs tend to use linear controllers as reference models for their simplicity and well-understood robustness characteristics.  This is despite the fact that the applications of these linear controllers are applied to a non-linear dynamical system such as attitude control with varying dynamic pressure.  Adaptive Controllers are non-linear and may offer performance benefits to non-linear systems as seen in aforementioned aerospace applications.  However, non-linear controller's stability properties need to be evaluated.  The Lyapunov stability criteria offer methods of evaluating these controller's boundedness and robustness behavior.

Aleksandr Lyapunov was a Russian mathematician who's work was published in 1892 \cite{lyapunov1892general} concerning the behavior of non-linear systems close to equilibrium without having to rigorously find the unique solutions to difficult differential equations used to model the system.  His work was largely overlooked until the Cold War when aerospace solutions required a more rigorous approach to analyzing non-linear control robustness.  Modern non-linear control engineers extensively utilize Lyapunov's techniques to design and evaluate non-linear controllers.

\section{Lyapunov Stability Theory}

According to Lyapunov, the stability properties of a system can be classified as stable, asymptotically stable, and exponentially stable.

Given the following autonomous nonlinear system:

 \begin{equation}
\dot{x}(t)=f(x(t)), \qquad x(0)=x_0
\end{equation}

where $f$ has equilibrium at $x_e$ :
 \begin{equation}
f(x_e) = 0
\end{equation}

then the equilibrium is said to be:
\begin{enumerate}
 \item \textbf{Lyapunov Stable} \newline
 for every $\epsilon > 0$ there exists a $\delta > 0$ such that, if \: $\norm{x(0) - x_e} < \delta$, then for every $t \geq 0$ we have  $ \norm{x(t) - x_e} < \epsilon$ 
 \item \textbf{Asymptotically Stable} \newline
 if the system is Lyapunov stable and there exists a $\delta > 0$ such that if \: $\norm{x(0) - x_e}  < \delta$, then $\displaystyle \lim_{t\to \infty} \norm{x(t)-x_e} =0$
 \item \textbf{Exponentially Stable} \newline
 if the system is asymptotically stable and there exists $\alpha > 0, \beta > 0, \delta > 0$ such that if $\norm{x(0)-x_e}<\delta$, then \:$\norm{x(t)-x_e} \leq \alpha \norm{x(0)-x_e} e^{-\beta t}$, for all $t \geq 0$
\end{enumerate}

Being Lyapunov stable infers that if a system is near equilibrium, then it will indefinitely remain near equilibrium.  If the system if found to be asymptotically stable then it eventually will achieve equilibrium as $t\to \infty$ and being exponentially stable implies it reaches equilibrium with a rate of convergence $\beta$.

\subsubsection{Lyapunov's Second Method}
 
Lyapunov's second method is also known as Lyapunov stability criteria.  This method offers a less tenuous method for evaluating mathematically non-ideal systems.  Lyapunov analysis of the linearized system around equilibrium can be cumbersome when the eigenvalues are purely imaginary.  In this case, the solutions can rapidly depart to infinity or approach zero with little perturbation to the eigenvalues.  Lyapunov's second method offers an alternative approach for classifying a system's stability using a concept that is similar to how total energy is defined in classical mechanics.

Conceptually, Lyapunov's second method can be compared to the evaluation of mechanical system similar to the modeling of energy in a vibrating spring mass system.  The energy of the unforced spring mass system will dissipate energy due to friction and or damping.  This trend of energy leaving the system towards some 'attractor' is evidence of the system's stability characteristics and identifies that there will be some stable end state.  Likewise, Lyapunov's second method specifies this with the use of a Lyapunov candidate function $V(x)$ which implicitly characterizes the total energy of the system.  It is important to note that Lyapunov realized that the candidate function could be any positive definite and radially unbounded function.  It is then said to be Lyapunov stable if any candidate function is found and meets the stability criteria.  This means that it is only incumbent upon the engineer to find one candidate function to meet the criteria.  This can be an iterative process of trying various energy like equations.  A common approach is to model the Lyapunov candidate equation as kinetic energy $(\frac{1}{2}u^2)$.  Lyapunov realized that characterizing the energy of a nonlinear system could be almost impossible for some cases, but this approach could prove stability without the rigorous knowledge of the true system's energy.

Lyapunov's second method defines a system as Lyapunov Stable for a system $\dot{x}=f(x)$ having an equilibrium point at $x=0$ where the Lyapunov candidate function $V(x):\mathbb{R}^n \rightarrow \mathbb{R}$ such that:
\begin{itemize}
 \item $V(x)=0$ if and only if $x=0$
 \item $V(x)>0$ if and only if $x\neq0$
 \item $\dot{V}(x)=\frac{d}{dt}V(x)=\sum\limits_{i=1}^{n} \frac{\partial V}{\partial x_i}f_i(x) \leq 0$, for all values of $x\neq 0$  
\end{itemize}

if $\dot{V}(x) < 0$ for $x\neq 0$ then system is asymptotically stable.

Additionally, it is required to demonstrate the condition of radial unboundedness to ensure the system is globally stable \cite{khalil1996noninear}.