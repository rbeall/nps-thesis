\chapter{Introduction and History}\label{ch:intro}

%This chapter outlines a brief history of adaptive control.  The motivation for adaptive control applied to fixed wing aircraft can be seen in this history as well as practical applications for modern aerodynamics.  Limitations of adaptive control will be discussed and ultimately explain why the \Lone adaptive control architecture was chosen for this research.  This will include the features specific to the \Lone approach which address some of these traditional limitations as well as a basic derivation as applied to a small fixed wing \ac{UAS}.

\section{Problem Statement}
The \ac{UAS} has evolved tremendously over the past decade.  Miniature autopilots have gotten smaller and cheaper with more sensitive and redundant sensor packages largely due to the cellular phone industry accelerating \ac{MEMS} technology.  The ability to manufacture these autonomous systems at fractions of the cost enables the advancement in multiple cooperative UAV applications including swarming capability.  This ability to mass-produce large quantities of \ac{UAS}'s poses an interesting challenge.  Even though the price has gone down and the performance has gone up, there still exists a significant amount of man-hours dedicated to sensor calibration and autopilot control law configuration and tuning for best achievable performance..  Tuning the six to ten conventional \ac{PID} controllers for one airframe is not an insignificant task.  Imagine a swarming systems often use the same airframes potentially assembled by the same manufacturer and all aircraft still require a tedious quality assurance check.  Physical aspects of the airframes such as \ac{CG}, control surface deflection/calibration/speed, airframe alignment, etc. all can drastically vary within the same delivered batch of airframes.  It should also be considered that most of these miniature \ac{UAS}'s experience hard landings, crashes, and/or damage in transportation, which all can effect aerodynamic handling qualities.  In summary, conventional control laws require a moderate to high level of expertise and require significant man-hours to tune properly for every airframe even if identical.  

\ac{UAS} avionics have drastically improved over the past decade, but the fundamental control law algorithms have not changed.  The \ac{PID} architecture found it's origin in automatic ship steering applications in 1922 \cite{minorsky1922pid}.  Conventional control law architectures for \ac{UAS}'s predominately still use \ac{PID} controllers.  Conventional control law architectures for \ac{UAS}'s predominately still use \ac{PID} controllers.  Their architecture offers a well understood and predictable behavior and for this reason is well suited for the aviation application.  The detriment of \ac{PID} control is that it application is mostly constrained by the use on a linear plant and most aerospace applications are non-linear and time varying.   An aircraft's control authority that increases proportionally to dynamic pressure is one example of aerodynamic non-linear control behavior.  In this case, the \ac{PID} controller's robustness to changes in velocity and/or density altitude is not guaranteed and for most aircraft has to be delicately handled with lookup tables produced from hours of flight test.



\section{Adaptive Control History}\label{history}
Adaptive control saw it's early debut in the NASA North American X-15 hypersonic rocket-powered X-plane experimental aircraft.  The X-15's performance envelope exceeded mach 6.0 and 300,000 feet \cite{jenkins2000x15specs}.  Engineers realized early on that the linear controllers performed well only at one dynamic pressure, but nowhere near the entire flight envelope.  Scheduling the controller gains with respect to dynamic pressure (gain scheduling) was one method used to help ensure robustness;  the method is still wide spread in commercial aviation due to its robustness but requires a lot of effort to 'explore' the entire flight envelope.  This was when the initial benefits of adaptive control were becoming realized.

The X-15 program started in 1959 and continued to 1968 flying nearly 200 successful flights.  It was considered one of NASA's most successful programs.  The benefit of adaptive control to the X-15 was that the adaptive controller was supposed to adjust the gain parameters online automatically.  If the controller was self tuning, it could potentially offer increased performance while reducing complexity.  The Honeywell MH-96 adaptive controller was implemented in the X-15-3 as a fly-by-wire controller designed to adaptively adjust the damping in pitch and roll with respect to a desired model response.  The goal was to achieve consistent aircraft response regardless of dynamic pressure and other variables.  During test flights of the MH-96 adaptive control, increased performance was observed especially in the dynamic phases of reentry over that of the linear fixed gain damping system \cite{dydek2010adaptive}.  These early breakthroughs in adaptive control proved the benefits could be viable aerospace solutions.  However, on November 15, 1967 there was a fatal accident caused by the adaptive controller.  The adaptive controller created an out of control flight situation resulting in dynamic pressures exceeding the structural limits and subsequent breakup of the airframe at 65,000 feet.

The turbulent start of adaptive control as implemented on the X-15 program was largely due to the early naive understanding of robustness.  Contemporary robust adaptive control strive to encapsulate these deficiencies of robustness in studies and proofs using Lyapunov stability analysis.  In addition to the developments of rigorous stability tools, a number of unique techniques have also been implemented to further increase controller robustness although.  One such technique utilizes dead band limits on the model adaptation process to avoid system/measurement noise from causing the un-learning of the states \cite{lavretsky2013robust}.  The \Lone adaptive control algorithm utilizes a technique which seeks to decouple the adaptation rate from robustness by 'low pass filtering' the contribution of the fast estimator under the premise that estimating the entire frequency spectrum is overly ambitious and should be limited to the bandwidth of the actuator.  Many advances have been made in the adaptive control field over the past few decades and this research sets out to evaluate a small subset of these techniques in the unforgiving aerospace environment.





