\label{appendix:system_identification}
\section{System Identification}

\subsection{Data Collection}
The Pixhawk autopilot was used to capture roll and pitch rates $(\dot{p},\dot{q})$ for the test vehicle as well as the pilot's command inputs.  These outputs and inputs were the essential building blocks for creating pitch rate and roll rate models for the test vehicle.  The autopilot is capable of logging data at 50-400 Hz that is represented as a discrete time domain signal.  This data should ultimately be manipulated into the s-domain.  The mathematics for this procedure are well defined, and numerous tools can be used to simplify this process \cite{tfest_matlab}.  

It is crucial to ensure there is sufficient frequency content in the data recorded.  Exciting multiple frequencies in the time domain ensures the regression techniques have an adequate persistence of excitation to resolve polynomial coefficients with higher certainty.  

To ensure sufficient frequency content was obtained from the aircraft, a linear chirp was chosen and implemented into the Pixhawk source code as follows:

\begin{equation}
x(t)=sin\left[\phi_0+2\pi\left(f_0t+\frac{k}{2}t^2\right)\right]
\end{equation}

where
\begin{itemize}
 \item[] $\phi_0$ is the initial phase of the chirp at t=0 (nominally zero degrees)
 \item[] $f_0$ is the initial frequency at t=0
 \item[] $k$ is the chirp rate
 \item[] $t$  is time in seconds
\end{itemize}

An example of this method can be seen in Figure~\ref{fig:chirp}.

\begin{figure}[!h]
 \centering
  \includegraphics[width=0.50\textwidth]{chirp.png}
  \caption{Reverse Linear Chirp Example}
  \label{fig:chirp}
\end{figure}

Specifically for this research, the reverse formulation of the linear chirp was used because it ensured the aircraft wouldn't exceed max roll or pitch limits within the first few cycles of the output.  The reverse chirp was applied in open loop while in \enquote{training mode}  which would constrain the angle of bank or pitch by switching into attitude hold mode upon an attitude exceedance.

\subsection{z-Domain to s-Domain}
The logged input and output data in discrete form requires shaping to convert cleanly into an s-domain representation.  The first step is ensuring that the data is of constant sampling rate.  In other words, the time between samples is uniform from sample to sample.  The data provided from the Pixhawk autopilot does not have a uniform sampling rate.  The sample rate is a user-defined rate (50-400Hz) but has a slight amount of jitter $(\pm 0.1\%)$.  \ac{PCHIP} interpolation was used to interpolate the data into a uniform sampling rate.

After the data is shaped correctly with a uniform sample rate, the discrete data is transformed into a continuous approximation using a \ac{ZOH} technique.  Taking the Laplace transform of the continuous input/output data will convert it into the s-domain, and finally, a non-linear least squares minimization algorithm can be run to find the polynomial coefficients which best fit the data.

The order of the regression (number of polynomials to estimate) is at the discretion of the engineer and their intuition of system's physical representation.  Higher order models will better fit the data, but in most cases, they tend to overfit the data; therefore some tradeoffs should be considered to simplify the model.  Most aircraft models reasonably limit the system to an \ac{LTI} and second-order.  These fundamental aerodynamics models divide the modeling into longitudinal and lateral dynamics.  Each axis of the aircraft is assigned two-second order responses.  Pitch, for example, has a second order response in the pitch damping mode (also known as the short period) and also has a second order response in the transition of kinetic energy to potential energy (also known as the long-period or phugoid).  Both first order and second order models were estimated for comparison sake.

Results were collected from two flight test events.  The first flight test was conducted under manual control without utilizing the chirp.  The pilot attempted to increase frequency of the input signal manually.  The second set of data collected was via the reverse linear chirp method previously described.  

The manually piloted acquired data was expected to have insufficient frequency content in the signal.  However, the manual flight test provided moderate results for modeling the aircraft as seen in Figure~\ref{fig:roll_model} .

\begin{figure}[!h]
 \centering
  \includegraphics[width=0.75\textwidth]{model_response.png}
  \caption{Roll Model Regression with Manual Inputs}
  \label{fig:roll_model}
\end{figure}

The results in Figure~\ref{fig:roll_model} demonstrates the utility of this technique even if data can only be acquired from manual pilot inputs.  It can be seen that the second order model starts to misrepresent the data at higher frequencies.  Figure~\ref{fig:chirp_model} illustrates the data recorded from the reverse chirp experiment run at 50Hz.  The $\delta$ \ac{PWM} input channel data clearly does not represent the software calculated chirp.  The highest frequency designed to be outputted during this experiment was 10Hz.  10 Hz is significantly less than the frequency required to meet Nyquist sampling criterion (25Hz)  for this data sampling rate.  However, there are clear patterns of aliasing in the input signal as recorded by the data flash logger.  

\begin{figure}[!h]
 \centering
  \includegraphics[width=0.75\textwidth]{chirp_response.png}
  \caption{Roll Model Regression with Reverse Linear Chirp}
  \label{fig:chirp_model}
\end{figure}

The chirp response was physically observed on pre-flight, in actual flight, and in the data-flash logged body rate of the aircraft.  However, the aliased input channel was arbitrarily biasing the regression result.  The significant aliasing in the logged input was not due to insufficient sampling rate as previously discussed.  After inspection,  the peculiar aliasing issue was hardware specific to the Pixhawk 1 autopilot in how the main CPU sends servo commands to the auxiliary I/O  CPU.  The most recent version of firmware, at the time of this test, improperly logs the \ac{PWM} through an aliased prone signal path.  The main and I/O CPU both run at 50Hz with some appreciable clock drift.  This generates a noticeable beat frequency and delay when the actual values in registry are saved for \ac{PWM} values are sent back round trip to the main CPU.  The implication of logging the \ac{PWM} values at the very end of the digital transmission line seems valuable in principle because the values being logged are the undeniable values being sent to the actuators.  However, the cost of logging these values in this manner on the Pixhawk architecture incurs significant aliasing at almost all frequencies.  Logging the commanded \ac{PWM} values before being sent to I/O CPU solved the aliasing discrepancy and produced very frequency rich models.

The manually piloted acquired data provided viable data source for the models even though it is a very simplistic approach.  There were two separate manual tests run on the same aircraft on the same flight, and the following are the results using this regression technique to model a second-order system:

\begin{equation}
H(s)=\frac{10.39}{s^2+31.26s+504.9}\
\end{equation}

and

\begin{equation}
H(s)=\frac{10.61}{s^2+29.77s+498.7}\
\end{equation}

Converting to standard form as described in Equation~\ref{eq:second_order_model} yields

\begin{equation}
H(s)=\frac{0.0206*22.47^2}{s^2+2*0.69*22.47s+22.47^2}\
\end{equation}

and

\begin{equation}
H(s)=\frac{0.0213*22.33^2}{s^2+2*0.67*22.33s+22.33^2}\
\end{equation}

It is important to note that this system identification technique run on separate sets of data has produced two models with very similar values for $\omega_n$ and $\zeta$.

This produces the average values of

$\omega_n=22.4 rad/s$ \newline
$k = 0.0209$ \newline
$\zeta=0.681$ \newline

With the aliasing removed from the chirped input command signals as previously described, the model is drastically improved and produces the following results:

\begin{equation}
H(s)=\frac{4.409}{s^2+27.11s+430.6}\
\end{equation}

and

\begin{equation}
H(s)=\frac{3.295}{s^2+18.82s+296.5}\
\end{equation}

Converting to standard form as described in Equation~\ref{eq:second_order_model} results in

\begin{equation}
H(s)=\frac{0.0102*20.75^2}{s^2+2*0.65*20.75s+20.75^2}\
\end{equation}

and

\begin{equation}
H(s)=\frac{0.0111*17.21^2}{s^2+2*0.54*17.21s+17.21^2}\
\end{equation}

The comparison of the model to recorded results is seen in Figure~\ref{fig:reverse_chirp_model}.

\begin{figure}[!h]
 \centering
  \includegraphics[width=0.75\textwidth]{reverse_chirp_roll.png}
  \caption{Non-Aliased Reverse Chirp Model Example}
  \label{fig:reverse_chirp_model}
\end{figure}

This produces the average values of

$\omega_n=18.98 rad/s$ \newline
$k = 0.010$ \newline
$\zeta=0.598$ \newline

In the author's experience, these values are reasonable values for this size and weight of airframe based on multiple system identification experiments in \ac{SITL} across multiple airframes as well as actual test flights of three different airframes.  This regression technique has shown potential to create realistic models from actual flight test data.  The data must be properly shaped.  The reverse chirp method has the potential to increase the fidelity of the high-frequency response of the aircraft if the aliasing issue can be resolved on the command input channel.