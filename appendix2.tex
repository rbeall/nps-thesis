%-------------------------------------------------------------------------------------
%-------------------Setup Matlab Listings syntax---------------------------
%-------------------------------------------------------------------------------------
\definecolor{mygreen}{RGB}{28,172,0} % color values Red, Green, Blue
\definecolor{mylilas}{RGB}{170,55,241}
\lstset{language=Matlab,%
    %basicstyle=\color{red},
    breaklines=true,%
    morekeywords={matlab2tikz},
    keywordstyle=\color{blue},%
    morekeywords=[2]{1}, keywordstyle=[2]{\color{black}},
    identifierstyle=\color{black},%
    stringstyle=\color{mylilas},
    commentstyle=\color{mygreen},%
    showstringspaces=false,%without this there will be a symbol in the places where there is a space
    numbers=left,%
    numberstyle={\tiny \color{black}},% size of the numbers
    numbersep=9pt, % this defines how far the numbers are from the text
    emph=[1]{for,end,break},emphstyle=[1]\color{red}, %some words to emphasise
    %emph=[2]{word1,word2}, emphstyle=[2]{style},    
}
%-------------------------------------------------------------------------------------
%-------------------------------------------------------------------------------------
\section{Euler vs Trapezoidal Method}
%-------------------------------------------------------------------------------------
\begin{lstlisting}[language=matlab]
clear, format long, clc, close all
dt = 1;

t0 = [0:.01:4];

y0 = exp(t0);

%Euler integration

t1 = [0:4];
y1 = exp(0);
for i=1:4
    y1(i+1)=y1(i) + dt*exp(i-1);
end

%Trapezoidal integration
t2 = [0:4];
y2 = exp(0);

for i=1:4
    y2(i+1)=y2(i) + dt*exp(i-1);
    y2(i+1)=y2(i) + (dt/2)*(exp(i-1)+y2(i+1));
end

plot(t0,y0,'k--',t1,y1,'b',t2,y2,'r')
legend('y=e^t','Euler Method','Trapezoidal Method')
\end{lstlisting}

\section{SISO Lyapunov Solution Proof}
%-------------------------------------------------------------------------------------
\begin{lstlisting}[language=matlab]
clear,clc, format compact, close all
%% Find Pb
wn = [5:0.1:10]; %rad/s
for i=1:length(wn)
    b = [wn(i)];
    a = [1,wn(i)];
    [A,B,C,D] = tf2ss(b,a)

    Pb(i) = lyap(A,1);
end
Pb_test = 1./(2*wn);

plot(wn,Pb_test,'o',wn,Pb)

%in this simple 1x1 matrix case Pb is easy to calc by hand in code
% Pb ends up being: Pb=1/2*wn;
\end{lstlisting}

\section{Reverse Linear Chirp}
%-------------------------------------------------------------------------------------
\begin{lstlisting}[language=matlab]
clear, format compact, clc, close all

fs = 1/200;
t = [0:fs:7];
f0= 0.01;%Hz
f1= 10; %Hz
k = (f1-f0)/(t(end));

phi_0 = 0.0;

sample_delay = 0.02;

out = sin(phi_0+2*pi*(f0*(7-(t+sample_delay))+(k/2).*(7-(t+sample_delay)).^2));
%out = 1500 + out*10;

out = out*10;

plot(t,out)
\end{lstlisting}

\section{Projection Operator Example Plots}\label{sec:projection_proof}
%-------------------------------------------------------------------------------------
\begin{lstlisting}[language=matlab]
clear, clc, format compact, close all

%% Plot Projection
epsilon = [0.25,0.28,0.3,0.38];
theta_max = 0.65;
theta_min = 0.25;
center = (theta_max+theta_min)/2;

span = (theta_max-theta_min)*1.4;
theta = [center-(span/2):0.01:center+(span/2)];

for j=1:length(epsilon)
    for i=1:length(theta)          
        %f_theta(i,j) = -(theta(i).^2-theta_max^2)/(epsilon(j)*theta_max^2);
        %f_theta_dot(i,j) = -(2*theta(i))/(epsilon(j)*theta_max^2);
        f_theta(i,j) = -(theta_min-theta(i))*(theta_max-theta(i))./(epsilon(j));
        f_theta_dot(i,j) = (theta_min+theta_max-(2*theta(i)))/(epsilon(j));
    end
end 

figure
for j=1:length(epsilon)   
hold on
plot(theta,f_theta(:,j))
end
line([min(theta),max(theta)],[0,0],'color','blue','linestyle','--')
line([theta_min,theta_min],[min(f_theta(:,1)),max(f_theta(:,1))],'color','red','linestyle','--')
line([theta_max,theta_max],[min(f_theta(:,1)),max(f_theta(:,1))],'color','red','linestyle','--')
legend('\epsilon=0.25','\epsilon=0.28','\epsilon=0.30','\epsilon=0.38')
ylabel('f(\theta)')
xlabel('\theta')
title('Projection Operator')
hold off
\end{lstlisting}
