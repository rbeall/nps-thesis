\chapter{Conclusion}\label{ch:conclusion}

The primary objective of this thesis was to evaluate a modern control technique to determine if the advanced controller could reduce the Navy's growing \ac{GNC} demand on engineering resources.  

The \Lone adaptive control algorithm suggested and developed in this thesis proved to be a successful alternative to the conventional \ac{PID} control strategy which drastically reduced the cost and time requirements to achieve robust flight.  In addition to meeting the objective of lowering the engineering demand of the aircraft \ac{GNC} algorithms, the \Lone adaptive control could also provide battle damage tolerance and achieve more accurate control with the utilization of fast and robust adaptation.

Conversion of the continuous time \Lone algorithm proved to be difficult with limited precedent literature but proved to be quite achievable with very little limitation posed by contemporary embedded processors.  The \ac{APM} flight stack open source project proved to be a very versatile software base that enable rapid prototyping and testing of the \Lone algorithm.  

The use of \ac{SITL} was a critical tool in the development of this algorithm into source code.  Follow on research or implementation of aircraft \ac{GNC} in the simulation environment drastically reduces the project risk at little to no expense using contemporary high fidelity aircraft simulation tools.

Flight test of the \Lone algorithm highlighted many capabilities gained through the use of fast and robust adaptation.  The primary goal of reducing the \ac{GNC} engineering demand required to achieve successful robust flight of \ac{UAS}'s was evident even on the first test flights.  Achieving adequate flight performance was now attainable within a matter of minutes instead of multiple tests spread across multiple flights potentially spread across multiple days.  With the use of \ac{SITL} integration, a robust model of the airframe, and more dedicated research, successful first flights with no tuning required are now within the realm of possibility.




