\documentclass[twoside,thesis,twoadvisorsreader]{npsreport}

\usepackage{doc,lipsum} % provides \BibTex and \lipsum macros, for demos
\usepackage{acronym}
\usepackage{commath}
\usepackage{amssymb}
\usepackage{amsmath}  %useful for multi-line equations

\usepackage{listings} %Used to post C++ code
\usepackage{xcolor}
\lstset { %
    language=C++,
    backgroundcolor=\color{black!5}, % set backgroundcolor
    basicstyle=\footnotesize,% basic font setting
}


\usepackage{graphicx}
\graphicspath{{figs/}}

\newcommand{\Lone}{$\mathcal{L}_1$ }
\newcommand{\degrees}{$^{\circ}$}

%
% For Example: you might find one of these useful:

%\usepackage{epstopdf}        % to use .eps files for figures
%\usepackage{bm}              % bold math if you need bold greek letters
%\usepackage{glossaries}      % see http:}/en.wikibooks.org/wiki/LaTeX/Glossary
%\usepackage{asymptote}       % for graphics
% The asymptote package allows for very nice graphics and figures
% Proper usage requires additional information located at:
% http://asymptote.sourceforge.net/
% See the gallery at this URL for examples

%\usepackage{placeins}        % float placement
% Provides \FloatBarrier which keeps figures/tables in the same section.
% LaTeX sometimes moves them to fill up pages.
% http://ftp.math.purdue.edu/mirrors/ctan.org/macros/latex/contrib/placeins/placeins-doc.pdf

%\usepackage[numbered]{mcode} % matlab code
% The mcode package must be separately downloaded.
% http://www.mathworks.com/matlabcentral/fileexchange/8015-m-code-latex-package

%\usepackage{flafter}         % float placement
% Ensures that figures/tables do not appear in the document before
% they are referenced in the text.


\title{Engineering of Fast and Robust Adaptive Control for Fixed-Wing Unmanned Aircraft}

% Student info
\author{Ryan G. Beall}
\rank{LT, USN}
\degree{Master of Science in Systems Engineering}
\degreeabbreviation{MS}   % Should be MS, MBA or MA
\prevdegrees{B.S., United States Naval Academy, 2008} % previous degree

% Department info
\department{Department of Systems Engineering}
\thesisadvisorone{Oleg A. Yakimenko}
\thesisadvisortwo{Vladimir N. Dobrokhodov}
\secondreader{Fotis A. Papoulias}
\departmentchair{Ronald E. Giachetti}

% The date you are graduating:
\degreedate{June 2017}

% See Thesis processor's release form for approved distribution statements.
\distribution{Approved for public release. Distribution is unlimited}

% Your abstract.  New paragraphs start after an empty line.
\abstract{
Due to the incraesing demand for autonomous unmanned systems, the Department of Defense (DOD) will soon have increased difficulty in managing costs and delievery schedule.  Unmanned Aerial System (UAS) technology has advanced significantly over the past decade accelerating the use of autonomous systems both in the DOD as well as the commerical sector.  As the demand for UAS technology increases, the current Guidance Navigation and Control (GNC) algorithms will prove to poorly scale to meet the demand.  Currently, significant resources are required to certifiy flight controllers on an individual platform basis.  As different airframes are introduced to meet the expanding mission requirements, the resorces required to sustain the GNC certification demand will become a limiting factor in scalability.\newline \newline
The feasibility of replacing conventional GNC techniques for stabilizing aircraft attitude with modern adaptive control theory was conducted on Commerical off the Shelf (COTS) opensource autopilots.  COTS autopilots and open source software enabled rapid prototyping and integration of an adaptive controller with limited resources.  It was determined that the adaptive control algorithm successfully stabilized aircraft attitude with improved performance over conventional controllers.  The adaptive controller architecture was designed specifically to be aircraft non-specific to ensure the controller easily scales to minimize the resource burden of tuning and certification.  The adaptive controller tested in this research improved performance over the baseline controller and was rapidly integrated on multiple various airframes with minimal resources.
}

% Switch the below lines around, if FOUO
\securitybanner{}  
%\securitybanner{FOR OFFICIAL USE ONLY}

%
% Mandatory fields for the SF298.
%
\ReportType{Master's Thesis}
\ReportDate{[Month Year]}       % for a thesis, graduation date
\SponsoringAgency{N/A}          % really, for technical reports
\DatesCovered{MM-DD-YYYY to MM-DD-YYYY}
\ReportClassification{Unclassified}
\AbstractClassification{Unclassified}
\PageClassification{Unclassified}
%
% Optional fields for the SF298.
%
\RPTpreparedFor{}
\ContractNumber{}
\GrantNumber{}
\ProgramElementNumber{}
\TaskNumber{}
\WorkUnitNumber{}
\POReportNumber{}
\Acronyms{}
\SMReportNumber{}
\SubjectTerms{}
\ResponsiblePerson{}
\RPTelephone{}
\SignatureOne{}
\SignatureTwo{}
\SupplementaryNotes{The views expressed in this document are those of
  the author and do not reflect the official policy or position of the
  Department of Defense or the U.S. Government. %
  IRB Protocol Number: N/A. % if you need to note an IRB Protocol or N/A
}

% Optional. Prevents footnotes from being reset at each chapter
% Comment this out to have them reset with each chapter.
\makeatletter
\@removefromreset{footnote}{chapter}
\makeatother

% Optional. Adds pdf metadata and links.
% This should be right before the \begin{document}, to be the
% last package / macros defined. (Hyper-ref is fragile,
% needs to be last, and has known conflicts with other packages.)
% Comment out if you have build problems building with hyperref
\NPShyperref

%
% Your thesis begins here
%
\begin{document}

\NPScover                  	% Cover page
%\NPSsftne                  		% SF298 form
%\NPSsignature             	% Tech Report page (iii): signature page
\NPSthesistitle            	% Thesis page (iii): title page
\NPSabstractpage           	% Abstract Page
\NPSfrontmatter            	% NPS front matter follows

% This changes the chaptermark and includes the various tables
% It must be here.
\renewcommand{\chaptermark}[1]{\markboth{\MakeUppercase{\chaptername}\ \thechapter.\ #1}{}}

%
% If you don't need one of these, comment it out.
%
\NPStableOfContents
\NPSlistOfFigures
%\NPSlistOfTables
\NPSlistOfAcronymsFromFile{acronyms}

\NPSexecsummary{
The field of adaptive control offers techniques for increasing performance and robustness in numerous settings and applications.  Adaptive control is different than traditional feedback in that it offers a mechanism for adjusting the controller's parameters to reduce plant uncertainty.  Traditional feedback control utilizes parameters, which are specified by the engineer to optimize an ideal use case, which often times requires extensive tuning and testing.  Adaptive controllers adjust their control parameters using various intelligent mechanisms designed to   increase robustness to plant variation or unanticipated disturbances.  Adaptive control has many applications in the aerospace domain to include control strategies when aerodynamic coefficients are unknown or are non-constant, actuator failure, airframe damage, etc.   This research evaluates fixed wing UAS controller performance and robustness using the \Lone adaptive control architecture. 
}

\NPSacknowledgements{
I would like to thank.....

}

% Start layout for the NPS body
\NPSbody


% CHAPTERS
\chapter{Introduction and Literature Review}\label{ch:intro}




 	%Introduction and History
\chapter{Background and Preliminaries}\label{ch:problem}
%---------------------------------------------------------

\section{Classical Feedback vs Adaptive Control}
Control of a system can be broken into two required elements.  There is the requirement to control the system from:
\begin{enumerate}
 \item disturbances which affect the controlled states 
 \item disturbances which affect the performance of the system as a whole
\end{enumerate}
Classical feedback control seeks to solve the first type of disturbance.  This form of control is meticulously tuned to achieve the desired overshoot and settling time for example.  The important assumption that is made by classical feedback controllers is that the underlying plant/system performance is not changing.  For example, the cruise control that maintains a vehicle's speed assumes that the available horsepower of the car is fixed.  This is a fairly good assumption as the horsepower with respect to rpm available at sea level and 5,000 feet for an internal combustion engine is constant enough that a fixed gain feedback controller would perform well at maintaining the speed of the vehicle in both environments.  In the case of an airplane, the dynamic pressure is proportional to velocity squared and can drastically change the performance of the aircraft.  In this case, the constant system performance assumption can cause a fixed gain classical feedback controller to go unstable at higher dynamic pressures (higher airspeeds).   Conversely, adaptive controllers assume that the system performance is unknown and is likely to vary with time.  Adaptive control seeks to ensure a system's performance with respect to characteristics, such as damping ratio and settling time, which are kept constant regardless of a plant's dynamics that may be unknown and time varying.  For both classical and adaptive control, there exists some form of error which drives the controller.  In the case of classical feedback, the error is calculated between the command and the feedback state of the plant. In adaptive control (in general), the error is calculated between the outputs of the desired reference model and real plant's measured performance.

Figure~\ref{fig:why_adaptive_control} outlines the decision making process a controls engineer makes when deciding the type of controller needed for a given circumstance.

%I would not spend too much time here. What critical is to demonstrate the fundamental difference. In the case of classical feedback, the error is calculated between the command and the feedback of the plant. In adaptive control (in general) the error is calculated between the outputs of the reference model and real plant.


\begin{figure}[h!]
 \centering
  \includegraphics[width=0.5\textwidth]{why_adaptive_control.png}
  \caption{Determine if adaptive control should be used}
  \label{fig:why_adaptive_control}
\end{figure}


%---------------------------------------------------------
\section{Model Reference Adaptive Control}
\ac{MRAC} establishes the foundation for most of modern, robust adaptive control.  Its structure is intuitive in nature and seeks to define a system's response to a command signal with a reference model.  Unlike traditional feedback where the error signal is generated with respect to state error, \ac{MRAC} attempts to achieve a system response with respect to some reference model performance. \ac{MRAC} assumes that there is some nominal response of the system which can be characterized with a model of unknown parameters.  The error between the model response and the system response generates the error for an 'adjustment mechanism' to learn the unknown model parameters.

\begin{figure}[h!]
 \centering
  \includegraphics[width=1.0\textwidth]{traditional_MRAC.png}
  \caption{Traditional \ac{MRAC} architecture }
  \label{fig:traditional_mrac}
\end{figure}

Figure~\ref{fig:traditional_mrac} illustrates a topology where a traditional feedback controller is established as an inner loop and the 'Reference Model' and 'Adjustment Mechanism' is established as an outer loop.  The outer loop attempts to minimize the error between the reference model output and the plant output.  Using this error to learn the system parameters can be done utilizing one of two methods: gradient descent or stability theory.

\subsection{MIT Rule - Gradient Decent}
One of the first approaches to \ac{MRAC} controllers was implemented at the Instrument Labs at MIT (now known as Draper Labs).  The gradient descent based method was called the 'MIT Rule' for this reason \cite{aastrom2013adaptive}.  This method attempts to learn some unknown parameter by descending the gradient of the error between the reference model and the plant output.

Given the simple first order system $G(s)$:
\begin{equation}
G(s) = k_{dc}\frac{1}{s+1}
\end{equation}

where $k_{dc}$ is some unknown feedforward gain.  In the case of the MIT rule, $k_{dc}$ is the parameter to be learned and is defined as $\theta$.  The first step in the MIT rule is to establish a cost (or loss) function.  One example of a cost function $J(\theta)$ is:
\begin{equation}
J(\theta) = \frac{1}{2}e^2
\end{equation}
\begin{equation}
e=y-y_m
\end{equation}

where $e$ is error, $y$ is the plant output, and $y_m$ is the model output.

In order for the cost function to be minimized, the negative gradient of the cost function is calculated and used to correct the a priori estimate.  This method takes the following form where $\gamma$ is the adaptation gain:
\begin{equation}
\frac{d\theta}{dt}=-\gamma \frac{\partial{J}}{\partial{\theta}} =-\gamma e\frac{\partial{e}}{\partial{\theta}}
\end{equation}

The stability of this method is very system dependent and heavily relies on trial and error to ensure the adaptation gain $(\gamma)$ is not too high.  This usually requires low adaptation rates for most systems and may not produce adequate results.  It should also be noted that this method presupposes there is adequate persistence of excitation.  Without a frequency rich error signal being generated by adequate persistence of excitation, the model will fail to adapt.  This method also offers no guarantees that the learned parameters will converge to their actual values.

\subsection{Lyapunov Stability Criteria}

Aerospace controllers tend to use linear controllers for their simplicity and well-understood robustness characteristics.  This is despite the fact that the applications of these linear controllers are applied to a non-linear dynamical system such as attitude control with varying dynamic pressure.  Adaptive Controllers are non-linear and may offer performance benefits to non-linear systems as seen in aforementioned aerospace applications.  However, non-linear controller's robustness properties need to be evaluated.  The Lyapunov stability criteria offer methods of evaluating these controller's boundedness and robustness behavior.

Aleksandr Lyapunov was a Russian mathematician who's work was published in 1892 \cite{lyapunov1892general} concerning the behavior of non-linear systems close to equilibrium without having to rigorously find the unique solutions to difficult differential equations used to model the system.  His work was largely overlooked until the Cold War when aerospace solutions required a more rigorous approach to analyzing non-linear control robustness.  Modern non-linear control engineers extensively utilize Lyapunov's techniques to design and evaluate non-linear controllers.

\subsubsection{Lyapunov Stability Definitions}

Lyapunov's methods attempt to evaluate autonomous nonlinear dynamical systems within the bounds of three classifications.  In this case, the autonomous system is defined as definable set of ordinary differential equations which are not explicitly dependent upon the independent variable.  These classifications can be used to define a nonlinear system as Lyapunov stable, asymptotically stable, or exponentially stable.

Given the following autonomous nonlinear dynamical system:

 \begin{equation}
\dot{x}(t)=f(x(t)), \qquad x(0)=x_0
\end{equation}

where $f$ has equilibrium at $x_e$ :
 \begin{equation}
f(x_e) = 0
\end{equation}

then the equilibrium is said to be:
\begin{enumerate}
 \item \textbf{Lyapunov Stable} \newline
 for every $\epsilon > 0$ there exists a $\delta > 0$ such that, if \: $\norm{x(0) - x_e} < \delta$, then for every $t \geq 0$ we have  $ \norm{x(t) - x_e} < \epsilon$ 
 \item \textbf{Asymptotically Stable} \newline
 if the system is Lyapunov stable and there exists a $\delta > 0$ such that if \: $\norm{x(0) - x_e}  < \delta$, then $\displaystyle \lim_{t\to \infty} \norm{x(t)-x_e} =0$
 \item \textbf{Exponentially Stable} \newline
 if the system is asymptotically stable and there exists $\alpha > 0, \beta > 0, \delta > 0$ such that if $\norm{x(0)-x_e}<\delta$, then \:$\norm{x(t)-x_e} \leq \alpha \norm{x(0)-x_e} e^{-\beta t}$, for all $t \geq 0$
\end{enumerate}

Being Lyapunov stable infers that if a system is near equilibrium, then it will indefinitely remain near equilibrium.  If the system if found to be asymptotically stable then it eventually will achieve equilibrium as $t\to \infty$ and being exponentially stable implies it reaches equilibrium even faster.

\subsubsection{Lyapunov's Second Method}
 
Lyapunov's second proposed method is also known as Lyapunov stability criteria.  This method offers a less tenuous method for evaluating mathematically non-ideal systems.  Lyapunov analysis of the linearized system around equilibrium can be cumbersome in the case where equilibrium is at the origin, or the eigenvalues are purely imaginary.  In this case, the solutions can rapidly depart to infinity or approach zero with little perturbation to the eigenvalues.  Lyapunov's second method offers an alternative approach for classifying a system's stability using a concept that is similar to how energy is defined in classical dynamics.

Conceptually, Lyapunov's second method can be compared to evaluating the energy of a vibrating spring mass system.  The energy of the unforced spring mass system will dissipate energy due to friction and or damping etc.  This trend of energy leaving the system towards some 'attractor' is evidence of the system's stability characteristics and identifies that there will be some stable end state.  Likewise, Lyapunov's second method characterizes this with the use of a Lyapunov candidate function $V(x)$.  It is important to note that Lyapunov realized that the candidate function could be any function so as long as one candidate function is found in agreement with the stability criteria.  It is then said to be Lyapunov stable if any candidate equation is found and meets the stability criteria.  This means that it is only incumbent upon the engineer to find one candidate equation to meet the criteria.  This can be an iterative process of trying various energy like equations.  A common approach is to model the Lyapunov candidate equation as kinetic energy $(\frac{1}{2}u^2)$.  Lyapunov realized that characterizing the energy of a nonlinear system could be almost impossible for some cases, but this approach could prove stability without the rigorous knowledge of the true system's energy.

Lyapunov's second method defines a system as Lyapunov Stable for a system $\dot{x}=f(x)$ having an equilibrium point at $x=0$ where the Lyapunov candidate function $V(x):\mathbb{R}^n \rightarrow \mathbb{R}$ such that:
\begin{itemize}
 \item $V(x)=0$ if and only if $x=0$
 \item $V(x)>0$ if and only if $x\neq0$
 \item $\dot{V}(x)=\frac{d}{dt}V(x)=\sum\limits_{i=1}^{n} \frac{\partial V}{\partial x_i}f_i(x) \leq 0$, for all values of $x\neq 0$  
\end{itemize}

if $\dot{V}(x) < 0$ for $x\neq 0$ then system is asymptotically stable.

Additionally, it is required to demonstrate the condition of radial unboundedness to ensure the system is globally stable.














 	%Background and Literature Review
\chapter{\Lone Adaptive Control Derivation}\label{ch:derivation}

Introduction here!!!

\section{\Lone Adaptive Control}
The \Lone adaptive controller is an evolution of the concepts implemented by \ac{MRAC}.  They are similar approaches designed to model a \ac{LTI} system with unknown constant parameters.  These parameters are adjusted to achieve the desired outcome of the error between the actual plant (system) and the referenced system model (state predictor) to asymptotically approach zero.   Adaptive control attempts to estimate the plant's unknown parameters in situ.  Parameter estimation is done using either direct or indirect architectures.  The indirect architecture attempts to estimate the system's parameters, which could be considered similar to  system identification.  Alternately the easier to implement direct architecture estimates the controller parameters explicitly.  These architectures can be seen below in Figures~\ref{fig:direct_mrac} and \ref{fig:indirect_mrac}.

\begin{figure}[h!]
 \centering
  \includegraphics[width=0.65\textwidth]{Direct_MRAC.png}
  \caption{Direct \ac{MRAC} architecture }
  \label{fig:direct_mrac}
\end{figure}

\begin{figure}[h!]
 \centering
  \includegraphics[width=0.65\textwidth]{Indirect_MRAC.png}
  \caption{Indirect \ac{MRAC} architecture }
  \label{fig:indirect_mrac}
\end{figure}

The \Lone adaptive control algorithm asserts that trying to estimate the plant uncertainties outside of the control actuators' bandwidth is overly ambitious.  The system's actuator bandwidth and the slow dynamics of the plant are most commonly the system's limiting factors, and the estimator's robustness/stability could be in question if un-modeled high frequency content exists in the plant.  % See RHORs example here? 
The \Lone adaptive control constrains the objective function by using a low-pass filter (first or second order) to band the frequency response in order to meet robustness specifications.  This low-pass filter should be tuned to a frequency response commensurate with the actuator's frequency response.  When looking at examples of where to place the low-pass filter in the direct and indirect architectures, it becomes clear that the indirect architecture is the only candidate.  Figures~\ref{fig:direct_mrac_lowpass} and \ref{fig:indirect_mrac_lowpass} illustrate the placement of the low-pass filter and its implication on the closed loop model. 

\begin{figure}[h!]
 \centering
  \includegraphics[width=0.65\textwidth]{Direct_MRAC_lowpass.png}
  \caption{Direct \ac{MRAC} architecture with low-pass filter }
  \label{fig:direct_mrac_lowpass}
\end{figure}

\begin{figure}[h!]
 \centering
  \includegraphics[width=0.65\textwidth]{Indirect_MRAC_lowpass.png}
  \caption{Indirect \ac{MRAC} architecture with low-pass filter }
  \label{fig:indirect_mrac_lowpass}
\end{figure}

 It can be seen that the low-pass filter in the direct architecture inherently changes the structure of the model with the cascading of the low-pass filter and plant block diagrams.  This change mathematically is not mirrored in the state predictor and therefore is not subtractable.  However, in the indirect case, the structure of the model is kept intact and the low-pass filter is applied to both the plant and the state predictor.  This ensures that the low-pass filter is subtractable when calculating the error state and the model's structure is kept intact.

In the primary literature for this research \cite{hovakimyan2010l1}, the author often refers to the state predictor as the reference model or companion model for the direct and indirect architectures respectively.  The reference model (direct architecture) intuitively maps the desired model response to the error feedback.  In the indirect architecture case, the error state is a result of the companion model plus the low-pass filter.  This subtle distinction is necessary because it must be accounted for when tuning the companion model with the included low-pass filter.

Many slight variations of the \Lone adaptive architectures have been derived for various use cases \cite{hovakimyan2010l1}.  Some of the following forms were studied for viability in the fixed wing \ac{UAS} use case:
\begin{itemize}
	\item \ac{SISO} with constant but unknown state parameters
	\item \ac{SISO} with time variant and/or nonlinear unknown state parameters
	\item \ac{MIMO} with constant but unknown state parameters
	\item \ac{MIMO} with time variant and/or nonlinear unknown state parameters
\end{itemize}

\ac{MIMO} control algorithms would potentially afford the controller more ability to cope with system coupling if present.  A fixed wing \ac{UAS} would exhibit coupled behavior due to the coupling present in the aerodynamics but was not chosen due to the added architectural complexity.  Unknown state parameters that are assumed to be constant or time invariant are considered matched uncertainty.  Unknown state parameters that are non-constant (time variant) and/or exhibit non-linear behavior are considered unmatched uncertainty.  The unmatched uncertainty architecture offers a more appealing solution for fixed wing use cases (asymmetric actuator failure, aerodynamic coefficients scaled by dynamic pressure, etc.), but adds a significant amount of complexity to the architecture.  In summary, the \ac{SISO} architecture with matched uncertainty was chosen for this research.  

The \ac{SISO} controller with matched uncertainty was chosen to control pitch rate $(q)$ and roll rate $(p)$ of the aircraft using two separate but parallel controllers.  This meant that the controller could be generalized to a first principles physical point mass model similar to derivations found in rigid body equations of motion.  In this implementation of the \Lone adaptive controller, the desired state $x$ to be controlled was an individual body rate (\eg $q$, $p$). 

\begin{figure}[h!]
 \centering
  \includegraphics[width=1.0\textwidth]{L1_architecture.png}
  \caption{\Lone Architecture with Matched Uncertainty Block Diagram \cite{hovakimyan2010l1} }
  \label{fig:l1_architecture}
\end{figure}

As seen in Figure~\ref{fig:l1_architecture}, the generalized \Lone architecture in block diagram form and the following elements can be identified:
\begin{itemize}
	\item[] $k_g$ - feed forward input gain
	\item[] $kD(s)$ - user described filter (second order low pass plus integrator)
	\item[] $\hat{\eta}$ - \Lone controller state
	\item[] $\dot{x}$ - first order differential equation of state model
	\item[] $\hat{x}$ - state estimate
	\item[] $\tilde{x}$ - state error
	\item[] $u$ - reference objective
	\item[] $A_m$ - Hurwitz matrix
	\item[] $b$ - input matrix
	\item[] $\hat{\omega}$ - unknown input gain coefficient
	\item[] $\hat{\theta}$ - unknown constant state coefficient
	\item[] $\hat{\sigma}$ - unknown disturbance estimate
	\item[] $\Gamma$ - adaptation gain
	\item[] $Pb$ - solution to the Lyapunov stability criterion	
\end{itemize}

It should also be noted that the architecture presented in Figure~\ref{fig:l1_architecture} includes the use of a projection operator.  The parameters for $\dot{\hat{\omega}}$, $\dot{\hat{\theta}}$, and $\dot{\hat{\sigma}}$ are all projection based adaptation laws.  This simply ensures that the adaptation stays bounded around the feasible region of parameter space.  The Lyapunov stability proofs for this architecture rely on this method to guarantee stability\cite{hovakimyan2010l1} .  More discussion on the specific application of this operator can be found in Appendix [???].

One of the main benefits of using the \ac{SISO} architecture is that the solution to the Lyapunov stability criterion ($Pb$) used in the projection based adaptation laws is greatly simplified.  

In this case, $Pb$ reduces to:
\begin{equation}
Pb = \frac{1}{2\omega_n}
\end{equation}

where $\omega_n$ is the natural frequency in rad/s for the first order companion model in discrete recursive form assuming DC gain of 1. 


%---------------------------------------------------
\section{\Lone Discrete Time Implementation}
Implementing any algorithm on actual autopilot hardware will inevitably force some if not all parts of the algorithm to be discretized.  Autopilots like the Pixhawk operate at some scheduled loop rate for executing the litany of subprograms that measure sensors, calculate navigation commands, and many more.  In the case of the Pixhawk autopilot, you can run the main loop up to 400 Hz.  At 400 Hz, there is a significant insurance that the vehicle's full frequency domain of importance will be achievable.  However, the \ac{APM} flight stack records all logged parameters also at this loop rate and can create log files larger than are reasonably desired.  There are a myriad of other reasons why the engineer would not want to run at high loop rates, but successful flight at the lowest (default of 50Hz) is desired if adequate performance of the adaptive control can be achieved.  Failures in early adaptive control were largely impart due to a very naive understanding of robustness.  Brian Anderson concludes that "it is clear that the identification time scale needs to be faster than the plant variation time scale, else identification cannot keep up" \cite{anderson2005failures}.   

\subsection{Digital Bi-Quad Filter}

The L1 adaptive control algorithm utilizes two specific elements that will require careful discretization; the companion model and the low pass filter.  The digital bi-quad filter offers a very versatile and straight forward method for accurately implementing the companion model and the low pass filter discretely using it's recursive nature.  It is a second order filter which uses a \ac{FIR} front end and an \ac{IIR} back end requiring 4 total memory blocks.  This topology allows the designer to create numerous types of filters (low pass, high pass, bandpass, etc) simply by choosing appropriate coefficients.  If a first order filter is needed then the higher order FIR/IIR terms can be set to zero.  Figure~\ref{fig:bi-quad} illustrates this filter's topology where the \ac{FIR} structure is the left two memory blocks and the \ac{IIR} structure is the right two memory blocks.

\begin{figure}[h!]
 \centering
  \includegraphics[width=1.0\textwidth]{bi-quad_filter.png}
  \caption{Digital Bi-quad Filter Architecture }
  \label{fig:bi-quad}
\end{figure}

To determine the structure of the coefficients, a bi-linear Z transform is used to convert a desired S-domain (continuous time domain) filter/model into the Z-domain (discrete time domain).  

\begin{figure}[h!]
 \centering
  \includegraphics[width=1.0\textwidth]{bi-linear_transform.png}
  \caption{Bi-linear Transform}
  \label{fig:bi-linear_transform}
\end{figure}

This derivation can be seen below for the second order low pass model:

\begin{equation}
	H(s) = \frac{1}{s^2+\frac{s}{Q}+1}
\end{equation}

where the bi-linear transform converts s to z via:

\begin{equation}
	s = \left(\frac{1}{K}\right)\left(\frac{z-1}{z+1}\right)
\end{equation}

$K$ is the 'pre-warping' factor which accounts for the transition of the vertical s-plane into the circular z-plane as seen in figure~\ref{fig:bi-linear_transform}.

where $\omega T$ is:

\begin{equation}
	\omega T = 2\pi\left(\frac{F_c}{F_s}\right)
\end{equation}

\begin{equation}
\begin{split}
	K &= tan\left(\frac{\omega T}{2}\right) \\
	&= tan\left(\pi\frac{F_c}{F_s}\right)
\end{split}
\end{equation}



$F_c$ is the desired corner frequency of the filter and $F_s$ is the sampling rate (or loop rate of the autopilot).
This 'pre-warping' is critical to ensure that the continuous time cutoff frequency desired is correctly established in the discrete implementation.  It is the engineer's discretion if pre-warping is required for the appropriate application, but the general guidance is to pre-warp the Z-domain coefficients if the desired cut-off frequency is close to Nyquist.  It was chosen for this application to always pre-warp the coefficients even though the error is small for corner frequencies which are fairly distant from Nyquist.  This was chosen simply because calculating the $tan()$ function real time on the CPU adds negligible computational strain but offers ease of tuning for the engineer.

Applying the bi-linear transform to the continuous time second order low pass filter results in:

\begin{equation}\label{eq:bi-linear}
	H(z) = \frac{1}{ \left[\left(\frac{1}{K}\right)\left(\frac{z-1}{z+1}\right)\right]^2+\frac{ \left(\frac{1}{K}\right)\left(\frac{z-1}{z+1}\right)}{Q}+1}
\end{equation}

The desired form is:

\begin{equation}\label{eq:bi-quad}
	H(z) = \frac{b_0 + b_1 z^{-1} + b_2 z^{-2}}{a_0 + a_1 z^{-1} + a_2 z^{-2}}
\end{equation}

Reducing equation~\ref{eq:bi-linear} to match the form in equation~\ref{eq:bi-quad} results in the following coefficients:

\begin{equation}
\begin{split}
	a_0 &= 1 \\
	a_1 &= \frac{2(K^2-1)}{K^2+\frac{K}{Q}+1} \\
	a_2 &= \frac{K^2-\frac{K}{Q}+1}{K^2+\frac{K}{Q}+1} \\
	b_0 &= \frac{K^2}{K^2+\frac{K}{Q}+1} \\
	b_1 &= 2b_0 \\
	b_2 &= b_0 	
\end{split}
\end{equation}

The bandwidth of the filter $Q$ can be set by the engineer.  For example, if the pass-band of the filter is desired to be flat (Butterworth) then $Q$ can be set equal to $\frac{1}{\sqrt{2}}$.  For this research the following C++ code segments were used to explicitly calculate the bi-quad low-pass filter implementation: \newline

\begin{lstlisting}
void DigitalBiquadFilter<T>::compute_params(float sample_freq, 
float cutoff_freq, biquad_params &ret) {
    ret.cutoff_freq = cutoff_freq;
    ret.sample_freq = sample_freq;

    float fr = sample_freq/cutoff_freq;
    float K = tanf(M_PI/fr);  //Pre-Warp calculation
    float c = 1.0f+2.0f*cosf(M_PI/4.0f)*K + K*K;

    ret.b0 = K*K/c;
    ret.b1 = 2.0f*ret.b0;
    ret.b2 = ret.b0;
    ret.a1 = 2.0f*(K*K-1.0f)/c;
    ret.a2 = (1.0f-2.0f*cosf(M_PI/4.0f)*K+K*K)/c;
}
\end{lstlisting}

\begin{lstlisting}
T DigitalBiquadFilter<T>::apply(const T &sample, 
const struct biquad_params &params) {
    
    T delay_element_0 = sample - _delay_element_1 * params.a1 
	    - _delay_element_2 * params.a2;
    
    T output = delay_element_0 * params.b0 
	    + _delay_element_1 * params.b1 
	    + _delay_element_2 * params.b2;

    _delay_element_2 = _delay_element_1;
    _delay_element_1 = delay_element_0;

    return output;
}

\end{lstlisting}

This implementation can be used as the \Lone low-pass filter and as the companion model.  It can be seen in the above code segment that $K$, the pre-warp factor, is explicitly calculated every iteration.

\subsection{Simplified Bi-quad First Order Model}

In the case of the companion model, a first order response may be desired.  As described in equations~\ref{eq:first_order_model} and \ref{eq:state_space_model}, the discrete first order model can be derived from a simplified Bi-quad as seen below in figure~\ref{fig:bi-quad_first_order}.  It can be seen that the first coefficient of the \ac{IIR} filter is kept from this topology.
\begin{figure}[h!]
 \centering
  \includegraphics[width=1.0\textwidth]{bi-quad_first_order.png}
  \caption{Digital Bi-quad Simplified First Order Low-pass Filter }
  \label{fig:bi-quad_first_order}
\end{figure}

The first order model can be specified by either its time constant (time in seconds to reach 63\% of steady state) or its -3dB corner frequency.  The system takes the form as seen in equation~\ref{eq:first_order_corner_model} when defined by its corner frequency.

\begin{equation}\label{eq:first_order_corner_model}
H(s)=\frac{\omega_n}{s+\omega_n}
\end{equation}

therefore the explicit calculation of the Bi-quad coefficients in this case becomes:

\begin{equation}\label{eq:first_order_coeffieicnts}
\begin{split}
	a_1&=e^{\left(\frac{-\omega_n}{F_s}\right)}  \\
	b_0&=1-a_1
\end{split}
\end{equation}

where $\omega_n$ is the -3dB corner frequency in radians per second and $F_s$ is the sampling frequency in Hz.

Therefore the discrete recursive form of the first order model becomes:

\begin{equation}
y_{i+1}=a_1y_{i-1}+b_0y_i
\end{equation}

Another form commonly seen in software form which is designed to optimize for speed takes the form:

\begin{lstlisting}
float b_0=exp(-f_c/f_s);
float out+=(in-out)*b_0;
\end{lstlisting}

\subsection{Euler vs Trapezoid Rule}

The model estimate as well as the parameter estimates for the \Lone algorithm are both numerically estimated using discrete integration.  The Euler method is a numerical procedure for solving ordinary differential equations. The Euler method as applied to discrete integration, is the fundamental method for recursively integrating a digital signal.  The algorithm takes the form: \newline
where $h$ is the uniform step size,
\begin{equation}
y_{i+1}=y_i+hf(t_i,y_i)
\end{equation}

The recursive trapezoidal method (Heun's method) takes the form:
\begin{equation}\label{eq:trapezoidal_integration}
\begin{split}
\tilde{y}_{i+1}&=y_i+hf(t_i,y_i) \\
y_{i+1}&=y_i+\frac{h}{2}[f(t_i,y_i)+f(t_{i+1},\tilde{y}_{i+1})]
\end{split}
\end{equation}

Comparing the accuracy of the two numerical methods for discretely calculating the integral of $y=e^t$ can be seen in figure~\ref{fig:trapezoidal_integration}:

\begin{figure}[h!]
 \centering
  \includegraphics[width=0.75\textwidth]{trapezoidal_integration.png}
  \caption{Euler vs Trapezoidal Integration error}
  \label{fig:trapezoidal_integration}
\end{figure}

As seen in equation~\ref{eq:trapezoidal_integration}, the recursive trapezoidal integration method only adds one more line of complexity to the algorithm for a significant gain in accuracy and therefore will be the chosen method applied for all discrete numerical integration in this research.













	%L1 Adaptive Control Derivation
\chapter{Design of Experimental Platform}\label{ch:platform}

This chapter presents the \ac{COTS} autopilots used in this research as well as outlining the Linux toolchain used to conduct |ac{SITL} and \ac{GCS} operations.  This chapter also introduces the airframes that were prototyped for testing specific performance characteristics discussed in Chapter~\ref{ch:performance}.

\section{Pixhawk Autopilot}
The Pixhawk autopilot is a collaborative project among open-source engineers which resulted in a high-performance autopilot which is capable of controlling aircraft, ground vehicles, and many others.  The primary reason this autopilot was chosen was because of the vast amount of support in the developer community.  The Pixhawk 1 autopilot hardware is effectively obsolete at the time of this writing, but the open-source code base is extremely flexible and continues to be ported to new hardware as it becomes available.  This has been the case for various Raspberry Pi autopilots as well as the Pixhawk 2.  The Pixhawk autopilot operates using two flight stacks (code base/operating systems); the PX4 flight stack and the \ac{APM} flight stack.  This research was implemented on the \ac{APM} flight stack primarily because the author's familiarity with the developer team which offers unparalleled assistance to the academic community.  The \ac{APM} codebase also offers a litany of open-source tools such as a Linux based \ac{GCS}, \ac{SITL} simulator, log analysis tools, and an \ac{API} for the RealFlight 7.5 simulator (high fidelity airframe simulation for small aircraft).

\begin{figure}[h!]
 \centering
  \includegraphics[width=0.65\textwidth]{pixhawk_connectors.png}
  \caption{Pixhawk 1 Autopilot Connection Diagram  \cite{apm_org}}
  \label{fig:pixhawk_autopilot}
\end{figure}

\subsection{Key Features}
The Pixhawk 1 Autopilot has the following features as found on \cite{apm_org}:
\begin{itemize}
\item 168 MHz / 252 MIPS Cortex-M4F
\item 14 PWM / Servo outputs (8 with failsafe and manual override, 6 auxiliary, high-power compatible)
\item Abundant connectivity options for additional peripherals (UART, I2C, CAN)
\item Integrated backup system for in-flight recovery and manual override with dedicated processor and stand-alone power supply (fixed-wing use)
\item Backup system integrates mixing, providing consistent autopilot and manual override mixing modes (fixed wing use)
\item Redundant power supply inputs and automatic failover
\item External safety switch
\item Multicolor LED main visual indicator
\item High-power, multi-tone piezo audio indicator
\item microSD card for high-rate logging over extended periods of time
\end{itemize}

\subsection{Specifications}
The Pixhawk 1 Autopilot has the following specifications as found on \cite{apm_org}:
\subsubsection{Processor}
\begin{itemize}
\item 32bit STM32F427 Cortex M4 core with FPU
\item 168 MHz
\item 256 KB RAM
\item 2 MB Flash
\item 32 bit STM32F103 failsafe co-processor
\end{itemize}
\subsubsection{Sensors}
\begin{itemize}
\item ST Micro L3GD20H 16 bit gyroscope
\item ST Micro LSM303D 14 bit accelerometer / magnetometer
\item Invensense MPU 6000 3-axis accelerometer/gyroscope
\item MEAS MS5611 barometer
\end{itemize}
\subsubsection{Interfaces}
\begin{itemize}
\item 5x UART (serial ports), one high-power capable, 2x with HW flow control
\item 2x CAN (one with internal 3.3V transceiver, one on expansion connector)
\item Spektrum DSM / DSM2 / DSM-X® Satellite compatible input
\item Futaba S.BUS® compatible input and output
\item PPM sum signal input
\item RSSI (PWM or voltage) input
\item I2C
\item SPI
\item 3.3 and 6.6V ADC inputs
\item Internal microUSB port and external microUSB port extension
\end{itemize}

\section{Ground Control Station}

The \ac{GCS} used for this research was MAVproxy \cite{mavproxy}.  It is an open-source python based \ac{GCS} which provides flexible communication and command with any autopilot utilizing the MAVlink protocol \cite{mavlink}.  Even though MAVproxy is written in Python (\ac{OS} agnostics language), it was found to be cumbersome to operate the \ac{GCS} on any other platform other than Linux.  This is primarily because the source code updates quite rapidly to support new features and the core developer (Andrew Tridgell) exclusively utilizes MAVproxy in Linux.  A significant amount of external libraries are utilized which results in a moderate amount of compatibility debugging for other \ac{OS}'s if desired.

\subsection{MAVproxy Features}
The following are summaries which proved to be extremely useful for this research \cite{mavproxy_wiki}:
\begin{itemize}
\item command-line, console based application. Plugins included in MAVProxy provide a basic \ac{GUI}.
\item network capable and run over any number of computers.
\item portable; capable of running on any POSIX OS with Python, pyserial, and select() function calls, which means Linux, OS X, Windows, and others.
\item light-weight design; runs on small netbooks.
\item tab-completion of commands.
\end{itemize}

\section{Simulation}
The \ac{APM} environment offers three versions of \ac{SITL} simulations.  The lowest fidelity \ac{SITL} is provided by MAVproxy, which is a simple 6-degree of freedom kinematics model with no environment or actuator modeling.  This proved to be adequate for initial testing but resulted in poorly tuned algorithms when actual flight tests were conducted.  The MAVproxy simulator was used for basic code debugging but nothing else.

The second \ac{SITL} offered in the \ac{APM} environment is X-plane 10.  This is a much higher fidelity simulation, which includes actuator models and environmental modeling.  X-plane 10 is primarily used for simulating full-scale aircraft and therefore is difficult to find models, which accurately represent the dynamics of small fixed-wing \ac{UAS}.  The open-source community provided model called the \enquote{maxi-swift} was similar enough to the airframe in this research that it provided adequate \ac{SITL} modeling which ensured robust flight test.

The last \ac{SITL} simulation tested under the \ac{APM} environment was the RealFlight 7.5 \ac{API}.  The RealFlight \ac{RC} airplane simulator offers some of the industry's highest fidelity simulations for small aircraft.  This product requires an \ac{API} key to hook into MAVproxy.  This capability is not yet on the market as of the time of this research, but the \ac{APM} core developers were supportive of this research and ran multiple experiments with the RealFlight \ac{SITL} for early validation.  The screenshot in Figure~\ref{fig:realflight_sitl} was captured while conducting testing utilizing the RealFlight \ac{SITL}.

\begin{figure}[h!]
 \centering
  \includegraphics[width=0.65\textwidth]{realflight_simulator.png}
  \caption{High Fidelity RealFlight 7.5 \ac{SITL}}
  \label{fig:realflight_sitl}
\end{figure}


\section{Airframe}

The aircraft used for this research was the Flitetest Spear and the Flitetest Explorer \cite{flitetest}.  The Spear airframe was chosen for its endurance capability of greater than 45 minutes of flight time and its large capacity fuselage.  The flying-wing architecture keeps the actuation requirement to a minimum of two servos by utilizing an elevon configuration.

Figures~\ref{fig:spear} and \ref{fig:spear_carg} are example photos from the instructional build website \cite{flitetest}.

\begin{figure}[!h]
 \centering
  \includegraphics[width=0.65\textwidth]{spear.png}
  \caption{Spear Airframe \cite{flitetest}}
  \label{fig:spear}
\end{figure}

The large blunt nose provides adequate space for two 2,200 mAh (12.6volts) lithium polymer batteries wired in parallel.  The remaining cargo space was used for accommodating the Pixhawk autopilot.

\begin{figure}[!h]
 \centering
  \includegraphics[width=0.6\textwidth]{spear_cargo.png}
  \caption{Spear Cargo Capacity \cite{flitetest}}
  \label{fig:spear_cargo}
\end{figure}

This plane was constructed out of craft foam board.  The plans were downloaded from flitetest.com\cite{flitetest} and converted to CorelDraw vector files for use in a laser cutter.  These files were then cut out of four sheets of foam board using the laser cutter.  The wing halves were joined with standard box tape and hot glue.  This provided a cheap and rapid construction process which was achievable under four hours of build time.

\begin{figure}[!h]
 \centering
  \includegraphics[width=0.6\textwidth]{spear_build.jpg}
  \caption{Spear Build Process}
  \label{fig:spear_build}
\end{figure}

\subsection{Spear Specifications}
\begin{itemize}
 \item weight without battery: 1.45 lbs (658 g)
 \item center of gravity: 3 – 3.5” (76 – 89 mm) in front of firewall
 \item control surface throws: 16\degrees  deflection – Expo 30\%
 \item wingspan: 41 inches (1041 mm)
 \item motor: 425 sized, 1200 kv minimum
 \item prop: 9 x 4.5 CW (reverse) prop
 \item electronic speed control (ESC): 30 amp minimum
 \item battery: (2) 2200 mAH 12.6 volt minimum
 \item xervos: (2) 9 gram servos 
\end{itemize}

The Explorer airframe was chosen because it is a conventional airframe with highly coupled aerodynamics which can be configured in multiple different failure modes.  The sport wing provided in the plans was modified to have independently actuated flaps, and ailerons for a combination of software enabled failure modes.  This aircraft was also configured with a rudder for testing the lateral aerodynamics coupling effects on the adaptive controller.  

\subsection{Explorer Specifications}
\begin{itemize}
 \item weight without battery: 1.08 lbs (493 g)
 \item center of gravity: 2.25” (57 mm) from leading edge of wing
 \item control surface throws: 12\degrees  deflection – Expo 30\%
 \item wingspan: 57 inches (1447 mm)
 \item motor: 425 sized, 1000 kv minimum
 \item prop: 9 x 6 CW (reverse) prop
 \item ESC: 30 amp minimum
 \item battery: (1) 2200 mAH 12.6 volt minimum
 \item servos: (6) 9 gram servos 
\end{itemize}

\begin{figure}[!h]
 \centering
  \includegraphics[width=0.90\textwidth]{explorer.png}
  \caption{FliteTest Explorer \cite{flitetest}}
  \label{fig:explorer_parts}
\end{figure}

\begin{figure}[!h]
 \centering
  \includegraphics[width=0.85\textwidth]{explorer_parts.png}
  \caption{Explorer Cut Foamboard Parts}
  \label{fig:explorer_parts}
\end{figure}

\begin{figure}[!h]
 \centering
  \includegraphics[width=0.75\textwidth]{explorer_electronics.png}
  \caption{Pixhawk Autopilot Installed on Explorer}
  \label{fig:explorer_electronics}
\end{figure}

The previously discussed \ac{COTS} autopilot, Linux software tool chain for \ac{SITL} and \ac{GCS}, and prototype aircraft were utilized to present the results found in Chapter~\ref{ch:performance}.



 	%Design of Experimental Platform
\chapter{Flight Testing and Performance Evaluation}\label{ch:performance}




 	%Flight Testing and Performance Evaluation
\chapter{Recommendation}\label{ch:recomendations}
This chapter provides various recommendations pertinent to the integration of the \Lone adaptive control algorithm.  These recommendations may be more qualitative than quantitative, but provide experience based guidance for the engineer to better understand the nuances of the \Lone adaptive control algorithm.  Prior to this research, there was very little experientially based guidance on implementation of this algorithm which proved to be the largest barrier for implementing this modern controller.

\section{\Lone Adaptive Control Algorithm Tuning}\label{sec:tuning}
The primary goal of this research was to reduce the complexity of tuning expertise required to get an unknown airframe airborne successfully.  The amount of time required to get the adequate airframe performance using the \Lone algorithm is significantly reduced.  Three primary features need tuning: the adaptation gain, the controller filter, and the companion model.

\subsection{Tuning the Adaptation gain}
Tuning the adaptation gain is fairly intuitive.  The adaptation gain is a function of the loop frequency at which the filter is run so it may only need to be tuned for a given autopilot at a given loop rate and not for every specific aircraft.  Anecdotally, the adaptation gain of 10,000 was used on the Pixhawk1 autopilot running the scheduled loop rate at 100Hz across multiple aircraft without needing to modify.  The primary feedback to the user for tuning this gain resides in the desired movement of the estimated states.  The adaptation gain was set low (1-10) while watching the parameters $(\theta, \omega, \sigma)$ adapt.  The adaptation gain is slowly increased until the desired rate of adaptation as seen in the real time monitoring of the parameters is adequate.  Another approach for tuning the adaptation rate is to increase the gain until parameters start to oscillate between the bounds and then reduce it.  This bang-bang response in the parameter adaptation will show up in the performance of the controller as increased peaks away from zero error between desired and achieved.  In the case of this research, this noise can be hard to identify in the rate control itself and therefore is why monitoring the attitude error helps find the appropriate adaptation gain which is extremely high but still not injecting attitude error spikes.

\subsection{Tuning the Controller Filter}
The adaptation feedback gain $(k)$, as discussed in Section~\ref{sec:l1_filter} was by far the most influential gain to tune.  The simplicity of tuning the \Lone algorithm resides in the fact that the majority of lay users could adequately tune an airframe with this stand-alone gain.  Adequate values for this gain ranged from 0.3 for responsive aircraft and 2.0 for very sluggish aircraft.  This value was set for both the roll and pitch axis independently.  As seen in Equation~\ref{eq:l1_filter}, this value is establishing the cutoff frequency of the control filter that separates the bandwidth limited control channel output and the high bandwidth adaptation.  The default value assigned in the source code was 0.45, which proved to be a good starting point.  If the default value for $k$ is not correct, then the control channel will produce either low-frequency oscillations or high-frequency oscillations.  The low-frequency oscillations are produced because the bandwidth of the control is too low and there should be a perceptible lag between the desired state and the achieved state.  High-frequency oscillations occur when the control filter bandwidth is set higher than the plant's bandwidth, and the aircraft is incapable of achieving the desired rates.  Unlike \ac{PID} control, the \Lone control never exhibited unstable performance with incorrect gains.  The controller simply oscillates with extremely poor performance.  This was a remarkable feature because poorly tuned \ac{PID} gains can cause an aircraft to depart controlled flight rapidly. Whereas, the \Lone controller maintained bounded flight performance as the theory suggests.  

\subsection{Tuning the Companion Model Cutoff Frequency}
System identification was conducted as seen in Appendix~\ref{appendix:system_identification} in order to ascertain the bandwidth of various airframes and their actuators.  Figure~\ref{fig:bode_analysis} is an example of second-order models for two aircraft's roll dynamics compared to second-order models of \ac{RC} actuators (servos).  As to be expected, the bandwidth for the actuators is slightly higher than the airframe dynamics.  
\begin{figure}[h!]
 \centering
  \includegraphics[width=0.85\textwidth]{bode_analysis.png}
  \caption{Aircraft Frequency Analysis }
  \label{fig:bode_analysis}
\end{figure}

These rough approximations were used to then place the cutoff frequencies of the companion model with the expectation that the companion model cannot achieve higher bandwidths than that of the airframe.  Conservatively, the companion model cutoff frequency was typically set 2-5 $rad/s$ lower than the expected max performance of the airframe.  After the other algorithm gains are tuned, and satisfactory performance is achieved, the companion model cutoff frequency can then be increased to achieve higher performance.

\section{Improved Recursive Architecture}

The speed of adaptation and accuracy of the discretized \Lone algorithm is drastically improved with increased loop frequency.  The algorithm was written as one recursive architecture that updates at the Pixhawk's scheduled loop rate.  As previously discussed, the scheduled loop rate ideally should run at lower frequencies to prevent excessive log file size and added strain on the CPU.  However, the \Lone architecture only requires the adaptation loop be run at faster rates.  With this specific performance enhancement in mind, the \ac{APM} architecture could be modified to accept an independent loop specifically for the \Lone adaptation update.  The sensors measurements and \ac{EKF} update can run significantly lower with only increased performance of the algorithm.  The higher adaptation loop would enable higher adaptation gains $(\Gamma)$ and consequently produce faster adaptation of the system.

\section{Integrator Windup Issue}
The \Lone controller architecture exhibits a similar response to integrator wind up in \ac{PID} control when the aircraft is not flying.  If the controller was enabled before takeoff, the control surfaces would move to counteract any slight deviation from input to output attempting to zero the error. Because the aircraft was not flying, the controller would continue to increase the control output until the actuators reached saturation.  The baseline code was configured to ensure that the parameter estimates would not continue to integrate after saturation was reached.  This technique is standard practice when writing control laws that utilize actuators that have saturation limitations.  However, the anti-windup feature that this offers only occurs when the actuators are saturated.  This is inadequate for the takeoff scenario.  The flight surfaces quickly saturate while waiting for takeoff and cause a crash immediately upon takeoff.  It is standard practice to also disable the integrator action of controllers if it is known that the aircraft is not flying using any combination of airspeed, throttle position, \ac{IMU} estimated velocity, or \ac{GPS} ground speed.  In the case of the \Lone algorithm, the integrator utilized in $D(s)$ is essential to the entire architecture and presented challenges in how to use ad-hoc methods to prevent the integrator windup issue even though the \enquote{non-flying} state was calculable onboard the autopilot.  The initial experiments were to see if the adaptation rate was fast enough to un-learn the aircraft's saturated state, but there were no combinations of filter gains which resulted in learning rates adequate to ensure safe takeoffs.  Further research in this area is required to completely replace the \ac{PID} controller with the \Lone.  All takeoffs were conducted either in manual control or with the \ac{PID} controller enabled until safely airborne.

In summary, no one manual has been created for this type of controller's implementation and integration.  The recommendations provided may only apply to this specific implementation of the \Lone adaptive controller, but it offers guidance where none previously was articulated in contemporary literature.
 	%Recommendation
\chapter{Conclusion}\label{ch:conclusion}

The primary objective of this thesis was to evaluate the \Lone Adaptive Control algorithm, to determine if the controller could reduce the NAVY's growing \ac{GNC} demand on engineering resources.  The \Lone adaptive control algorithm proved to be a successful alternative to the conventional \ac{PID} control strategy which drastically reduced the cost and time requirements to achieve robust flight.  In addition to meeting the objective of lowering the engineering demand of the aircraft \ac{GNC} algorithms, the \Lone adaptive control also provided battle damage tolerance and achieved more accurate control with the utilization of fast and robust adaptation.

Conversion of the continuous time \Lone algorithm proved to be difficult with limited precedent literature but proved to be quite achievable with very little limitation posed by contemporary embedded processors.  The \ac{APM} flight stack open source project proved to be a very versatile software base that enable rapid prototyping and testing of the \Lone algorithm.  

The use of \ac{SITL} was a critical tool in the development of this algorithm into source code.  Follow on research or implementation of aircraft \ac{GNC} in the simulation environment drastically reduces the project risk at little to no expense using contemporary high fidelity aircraft simulation tools.

Flight test of the \Lone algorithm highlighted many capabilities gained through the use of fast and robust adaptation.  The primary goal of reducing the \ac{GNC} engineering demand required to achieve successful robust flight of \ac{UAS}'s was evident even on the first test flights.  Achieving adequate flight performance were now attainable within a matter of minutes instead of multiple tests spread across multiple flights potentially spread across multiple days.  With the use of \ac{SITL} integration, a robust model of the airframe, and more dedicated research, successful first flights with no tuning required are now within the realm of possibility.




 	%Conclusion


% APPENDICES
% You have two recommended options for your appendix:
% a) A single appendix (with a single TOC entry)
% b) Multiple appendices. Look under the examples directory for a demo of
%   multiple appendices.
%

\NPSappendices

\chapter{Transfer Functions}
\label{appendix:transfer_functions}
\section{Transfer Functions}

This research utilizes the \ac{TF} representation of aircraft flight dynamics that is typical of \ac{LTI} systems.  A transfer function is a very useful approach to describe the relationship between inputs and outputs of \ac{LTI} systems.  Both analytically and numerically, the \ac{TF} approach has significant benefits in continuous and discrete time domains as its construct is based on well-developed properties and primitives of polynomials.  These polynomial representations in the s or z domain map to aerodynamic and inertial coefficients through equations of motion.  What is unknown or partially known a priori, are the numerical values of coefficients for those polynomials. Therefore the tools from the areas of online estimation such as regression are utilized to solve for them.

Transfer functions take the form

\begin{equation}
H(s)=\frac{Y(s)}{X(s)}\
\end{equation}

where
\begin{itemize}
 \item[] Y(s) is the Laplace transform of the output
 \item[] X(s) is the Laplace transform of the input
\end{itemize}

Standard physics models of first and second order form are well understood and seen in many model derivations.  The first-order model takes the form

\begin{equation}\label{eq:first_order_model}
H(s)=\frac{k_{dc}}{\tau s+1}=\frac{\omega_n}{s+\omega_n}
\end{equation}

where
\begin{itemize}
 \item[] $k_{dc}$ is the DC gain
 \item[] $\tau$ is the system time constant (time in seconds to reach 63\% of steady state)
 \item[] $\omega_n$ is the natural frequency of the first order system
\end{itemize}

Similarly the standard form for a second-order system takes the form

\begin{equation} \label{eq:second_order_model}
H(s)=\frac{\omega_0^2}{s^2+2\zeta\omega_0s+\omega_0^2}\
\end{equation}

where
\begin{itemize}
 \item[] $\omega_0$ is the system natural frequency in radians per second
 \item[] $\zeta$ is the system damping ratio
\end{itemize}

The modeling of a system can also be converted to a system of first-order differential equations also known as state-space modeling.  In this case, the first-order system model can be represented as
\begin{equation}\label{eq:state_space_model}
\dot{x}(t)=Ax(t)+Bu(t)
\end{equation}
where $\dot{x}$ is the time derivative of the state, $A$ is the state transition matrix with all its eigenvalues chosen negative (Hurwitz), $B$ is the input matrix, and $u$ is the input vector.








\chapter{Fixed Wing Aircraft Dynamics Model}
\label{appendix:dynamics_model}
\section{Fixed Wing Aircraft Dynamics Model}

The following is the nomenclature that will be used to describe the kinematic equations.  Euler angles for pitch $(\theta)$, roll $(\phi)$, and yaw $(\psi)$ will have the units of radians.  The following Figure~\ref{fig:reference_frame} illustrates the \ac{NED} reference frame definitions used for body rotational rates about the $x$ axis $(p)$, $y$ axis $(q)$, and the $z$ axis $(r)$ as well as the body velocities in the $x$ axis $(u)$, $y$ axis $(v)$, and the $z$ axis $(w)$.

\begin{figure}[h!]
 \centering
  \includegraphics[width=0.65\textwidth]{body_frame_rotations.png}
  \caption{Reference frame - body rates and velocities}
  \label{fig:reference_frame}
\end{figure}

The primary goal of this research is to implement an algorithm which controls fixed wing aircraft attitude.  Therefore, the focus of the following kinematic and dynamics equations will primarily concentrate on deriving only rotational dynamics from first principles.  

Newton's second law as it pertains to rotational motion can be stated as
\begin{equation}\label{eq:rotational_inertia}
\tau=J\frac{d\omega}{dt_i}
\end{equation}

where $\tau$ is the torques applied to the body, $J$ is the moment of inertia, and $ \frac{d\omega}{dt_i}$ is the angular acceleration of the body with respect to the inertial frame.

Equation~\ref{eq:rotational_inertia} can be rewritten in the body reference frame as follows:

\begin{equation}
\tau^b=J\dot{\omega}_{b/i}^b+\omega_{b/i}^b\times\left(J\omega_{b/i}^b\right)
\end{equation}

The expression $\dot{\omega}_{b/i}^b$ is the angular acceleration in the body frame as viewed in the body frame:

\begin{equation}
\dot{\omega}_{b/i}^b=
\begin{pmatrix}
\dot{p}\\
\dot{q}\\
\dot{r}
\end{pmatrix}
\end{equation}

The equation can then be rewritten with respect to $\dot{\omega}_{b/i}^b$:

\begin{equation}
\dot{\omega}_{b/i}^b=J^{-1}\left[-\omega_{b/i}^b\times\left(J\omega_{b/i}^b\right)+\tau^b\right]
\end{equation}

$J$ can be defined as the inertia matrix as follows

\begin{equation}
J=
 \begin{pmatrix}
 J_x & -J_{xy} & -J_{xz}\\
 -J_{xy} & J_y & -J_{yz}\\
 -J_{xz} & -J_{yz} & J_z
 \end{pmatrix}
\end{equation}

The moments of inertia, or the diagonal terms, are always non-zero for any rigid body.  The products of inertia, or the off-diagonal terms, are terms which describe the inertial coupling between axis.  For a traditional fixed wing aircraft, the natural symmetry will simplify the inertia matrix in the off-diagonal terms as follows:

\begin{equation}
J=
 \begin{pmatrix}
 J_x & 0 & -J_{xz}\\
 0 & J_y & 0\\
 -J_{xz} & 0 & J_z
 \end{pmatrix}
\end{equation}

The inverse of $J$ can be found to be

\begin{equation}
J^{-1}=
 \begin{pmatrix}
 \frac{J_z}{\Gamma} & 0 & \frac{J_{xz}}{\Gamma}\\
 0 & \frac{1}{J_y} & 0\\
 \frac{J_{xz}}{\Gamma} & 0 & \frac{J_x}{\Gamma}
 \end{pmatrix}
\end{equation}

where,

\begin{equation}
\Gamma = J_xJ_z-J_{xz}^2
\end{equation}

Aircraft nomenclature for torques are defined $\tau\triangleq(l,m,n)^T$ and therefore the combined equations derived from first principles take the form:

\begin{equation}\label{eq:body_rate_derivation}
\begin{split}
 \begin{pmatrix}
  \dot{p} \\
  \dot{q} \\
  \dot{r} 
 \end{pmatrix}
 &=
 \begin{pmatrix}
 \frac{J_z}{\Gamma} & 0 & \frac{J_{xz}}{\Gamma}\\
 0 & \frac{1}{J_y} & 0\\
 \frac{J_{xz}}{\Gamma} & 0 & \frac{J_x}{\Gamma}
 \end{pmatrix}
 \left[
 \begin{pmatrix}
  0& r& -q \\
  -r& 0& p \\
  q& -p& 0
 \end{pmatrix}
 \begin{pmatrix}
 J_x & 0 & -J_{xz}\\
 0 & J_y & 0\\
 -J_{xz} & 0 & J_z
 \end{pmatrix}
 \begin{pmatrix}
  p\\
  q\\
  r
 \end{pmatrix} +
 \begin{pmatrix}
  l\\
  m\\
  n
 \end{pmatrix}
 \right] \\ 
 &=
 \begin{pmatrix}
  \frac{J_z}{\Gamma} & 0 & \frac{J_{xz}}{\Gamma}\\
  0 & \frac{1}{J_y} & 0\\
  \frac{J_{xz}}{\Gamma} & 0 & \frac{J_x}{\Gamma}
 \end{pmatrix}
 \left[
 \begin{pmatrix}
 J_{xz}pq+(J_y-J_z)qr\\
 J_{xz}(r^2-p^2)+(J_z-J_x)pr\\
 (J_x-J_y)pq-J_{xz}qr
 \end{pmatrix}+
 \begin{pmatrix}
  l\\
  m\\
  n
 \end{pmatrix}
 \right]\\ 
 &=
 \begin{pmatrix}
  \Gamma_1pq-\Gamma_2qr+\Gamma_3l+\Gamma_4n\\
  \Gamma_5pr-\Gamma_6(p^2-r^2)+\frac{1}{J_y}m\\
  \Gamma_7pq-\Gamma_1qr+\Gamma_4l+\Gamma_8n
 \end{pmatrix}
 \end{split}  
\end{equation}
where,

\begin{equation}
\begin{split}
\Gamma_1&=\frac{J_{xz}(J_x-J_y+J_z)}{\Gamma}\\
\Gamma_2&=\frac{J_z(J_z-J_y)+J_{xz}^2}{\Gamma}\\
\Gamma_3&=\frac{J_z}{\Gamma}\\
\Gamma_4&=\frac{J_{xz}}{\Gamma}\\
\Gamma_5&=\frac{J_z-J_x}{J_y}\\
\Gamma_6&=\frac{J_{xz}}{J_y}\\
\Gamma_7&=\frac{J_x(J_x-J_y)+J_{xz}^2}{\Gamma}\\
\Gamma_8&=\frac{J_x}{\Gamma}\\
\end{split}
\end{equation}

The aerodynamic torques (excluding propulsive torques) can be found to be:

\begin{equation}\label{eq:aero_torques}
\begin{pmatrix}
  l\\
  m\\
  n
 \end{pmatrix}
 =
 \frac{1}{2}\rho V_a^2S
 \begin{pmatrix}
  b\left[C_{l_0}+C_{l_\beta}\beta+C_{l_p}\frac{b}{2Va}p+C_{l_r}\frac{b}{2V_a}r+C_{l_{\delta_a}}\delta_a+C_{l_{\delta_r}}\delta_r\right]\\
  c\left[C_{m_0}+C_{m_\alpha}\alpha+C_{m_q}\frac{c}{2V_a}q+C_{m_{\delta_e}}\delta_e\right]\\
  b\left[C_{n_0}+C_{n_\beta}\beta+C_{n_p}\frac{b}{2Va}p+C_{n_r}\frac{b}{2V_a}r+C_{n_{\delta_a}}\delta_a+C_{n_{\delta_r}}\delta_r\right]
 \end{pmatrix}
\end{equation}

Substituting the aerodynamic torques found in equation~\ref{eq:aero_torques} into equation~\ref{eq:body_rate_derivation} results in \cite{beard2012small}:
\begin{equation}\label{eq:body_rate_equations}
\begin{split}
 \dot{p}&=\Gamma_1pq-\Gamma_2qr+\frac{1}{2}\rho V_a^2Sb\left[C_{p_0}+C_{p_\beta}\beta+C_{p_p}\frac{bp}{2V_a}+C_{p_r}\frac{br}{2V_a}+C_{p_{\delta_a}}\delta_a+C_{p_{\delta_r}}\delta_r\right]\\
 \dot{q}&=\Gamma_5pr-\Gamma_6(p^2-r^2)+\frac{1}{2}\rho V_a^2Sc\frac{1}{J_y}\left[C_{m_0}+C_{m_\alpha}\alpha+C_{m_q}\frac{cq}{2V_a}+C_{m_{\delta_e}}\delta_e\right]\\
 \dot{r}&=\Gamma_7pq-\Gamma_1qr+\frac{1}{2}\rho V_a^2Sb\left[C_{r_0}+C_{r_\beta}\beta+C_{r_p}\frac{bp}{2V_a}+C_{r_r}\frac{br}{2V_a}+C_{r_{\delta_a}}\delta_a+C_{r_{\delta_r}}\delta_r\right]
\end{split} 
\end{equation}

Simplifying equation~\ref{eq:body_rate_equations} assuming no inertial or aerodynamic coupling results in:

\begin{equation}\label{eq:body_rate_simplified}
\begin{split}
 \dot{p}&=\frac{1}{2}\rho V_a^2Sb\left[C_{p_{\delta_a}}\delta_a+C_{p_p}\frac{bp}{2V_a}+C_{p_0}\right]\\
 \dot{q}&=\frac{1}{2}\rho V_a^2Sc\frac{1}{J_y}\left[C_{m_{\delta_e}}\delta_e+C_{m_q}\frac{cq}{2V_a}+C_{m_0}\right]\\
 \dot{r}&=\frac{1}{2}\rho V_a^2Sb\left[C_{r_{\delta_r}}\delta_r+C_{r_r}\frac{br}{2V_a}+C_{r_0}\right]
\end{split} 
\end{equation}

The equations in equation~\ref{eq:body_rate_simplified} are then slightly modified to fit the first order \ac{ODE} model as described in equation~\ref{eq:state_space_model}.
\begin{equation}\label{eq:simplified_ac_model}
\begin{split}
\dot{p}&=A_p\hat{p}+b_p\left(\hat{\omega}_p\delta_a+\hat{\theta}_pp+\hat{\sigma}_p\right)\\
\dot{q}&=A_q\hat{q}+b_q\left(\hat{\omega}_q\delta_e+\hat{\theta}_qq+\hat{\sigma}_q\right)\\
\dot{r}&=A_r\hat{r}+b_r\left(\hat{\omega}_r\delta_r+\hat{\theta}_rr+\hat{\sigma}_r\right)
\end{split}
\end{equation}
where the parameters are as follows,
\begin{itemize}
 \item[] $\omega$ - input gain coefficient
 \item[] $\theta$ - constant state coefficient
 \item[] $\sigma$ - disturbance estimate
\end{itemize}



\chapter{System Identification}
\label{appendix:system_identification}
\section{System Identification}

\subsection{Data Collection}
The Pixhawk autopilot was used to capture roll and pitch rates $(\dot{p},\dot{q})$ for the test vehicle as well as the pilot's command inputs.  These outputs and inputs were the essential building blocks for creating pitch rate and roll rate models for the test vehicle.  The autopilot is capable of logging data at 50-400 Hz that is represented as a discrete time domain signal.  This data should ultimately be manipulated into the s-domain.  The mathematics for this procedure are well defined, and numerous tools can be used to simplify this process \cite{tfest_matlab}.  

It is crucial to ensure there is sufficient frequency content in the data recorded.  Exciting multiple frequencies in the time domain ensures the regression techniques have an adequate persistence of excitation to resolve polynomial coefficients with higher certainty.  

To ensure sufficient frequency content was obtained from the aircraft, a linear chirp was chosen and implemented into the Pixhawk source code as follows:

\begin{equation}
x(t)=sin\left[\phi_0+2\pi\left(f_0t+\frac{k}{2}t^2\right)\right]
\end{equation}

where
\begin{itemize}
 \item[] $\phi_0$ is the initial phase of the chirp at t=0 (nominally zero degrees)
 \item[] $f_0$ is the initial frequency at t=0
 \item[] $k$ is the chirp rate
 \item[] $t$  is time in seconds
\end{itemize}

An example of this method can be seen in Figure~\ref{fig:chirp}.

\begin{figure}[!h]
 \centering
  \includegraphics[width=0.50\textwidth]{chirp.png}
  \caption{Reverse Linear Chirp Example}
  \label{fig:chirp}
\end{figure}

Specifically for this research, the reverse formulation of the linear chirp was used because it ensured the aircraft wouldn't exceed max roll or pitch limits within the first few cycles of the output.  The reverse chirp was applied in open loop while in \enquote{training mode}  which would constrain the angle of bank or pitch by switching into attitude hold mode upon an attitude exceedance.

\subsection{z-Domain to s-Domain}
The logged input and output data in discrete form requires shaping to convert cleanly into an s-domain representation.  The first step is ensuring that the data is of constant sampling rate.  In other words, the time between samples is uniform from sample to sample.  The data provided from the Pixhawk autopilot does not have a uniform sampling rate.  The sample rate is a user-defined rate (50-400Hz) but has a slight amount of jitter $(\pm 0.1\%)$.  \ac{PCHIP} interpolation was used to interpolate the data into a uniform sampling rate.

After the data is shaped correctly with a uniform sample rate, the discrete data is transformed into a continuous approximation using a \ac{ZOH} technique.  Taking the Laplace transform of the continuous input/output data will convert it into the s-domain, and finally, a non-linear least squares minimization algorithm can be run to find the polynomial coefficients which best fit the data.

The order of the regression (number of polynomials to estimate) is at the discretion of the engineer and their intuition of system's physical representation.  Higher order models will better fit the data, but in most cases, they tend to overfit the data; therefore some tradeoffs should be considered to simplify the model.  Most aircraft models reasonably limit the system to an \ac{LTI} and second-order.  These fundamental aerodynamics models divide the modeling into longitudinal and lateral dynamics.  Each axis of the aircraft is assigned two-second order responses.  Pitch, for example, has a second order response in the pitch damping mode (also known as the short period) and also has a second order response in the transition of kinetic energy to potential energy (also known as the long-period or phugoid).  Both first order and second order models were estimated for comparison sake.

Results were collected from two flight test events.  The first flight test was conducted under manual control without utilizing the chirp.  The pilot attempted to increase frequency of the input signal manually.  The second set of data collected was via the reverse linear chirp method previously described.  

The manually piloted acquired data was expected to have insufficient frequency content in the signal.  However, the manual flight test provided moderate results for modeling the aircraft as seen in Figure~\ref{fig:roll_model} .

\begin{figure}[!h]
 \centering
  \includegraphics[width=0.75\textwidth]{model_response.png}
  \caption{Roll Model Regression with Manual Inputs}
  \label{fig:roll_model}
\end{figure}

The results in Figure~\ref{fig:roll_model} demonstrates the utility of this technique even if data can only be acquired from manual pilot inputs.  It can be seen that the second order model starts to misrepresent the data at higher frequencies.  Figure~\ref{fig:chirp_model} illustrates the data recorded from the reverse chirp experiment run at 50Hz.  The $\delta$ \ac{PWM} input channel data clearly does not represent the software calculated chirp.  The highest frequency designed to be outputted during this experiment was 10Hz.  10 Hz is significantly less than the frequency required to meet Nyquist sampling criterion (25Hz)  for this data sampling rate.  However, there are clear patterns of aliasing in the input signal as recorded by the data flash logger.  

\begin{figure}[!h]
 \centering
  \includegraphics[width=0.75\textwidth]{chirp_response.png}
  \caption{Roll Model Regression with Reverse Linear Chirp}
  \label{fig:chirp_model}
\end{figure}

The chirp response was physically observed on pre-flight, in actual flight, and in the data-flash logged body rate of the aircraft.  However, the aliased input channel was arbitrarily biasing the regression result.  The significant aliasing in the logged input was not due to insufficient sampling rate as previously discussed.  After inspection,  the peculiar aliasing issue was hardware specific to the Pixhawk 1 autopilot in how the main CPU sends servo commands to the auxiliary I/O  CPU.  The most recent version of firmware, at the time of this test, improperly logs the \ac{PWM} through an aliased prone signal path.  The main and I/O CPU both run at 50Hz with some appreciable clock drift.  This generates a noticeable beat frequency and delay when the actual values in registry are saved for \ac{PWM} values are sent back round trip to the main CPU.  The implication of logging the \ac{PWM} values at the very end of the digital transmission line seems valuable in principle because the values being logged are the undeniable values being sent to the actuators.  However, the cost of logging these values in this manner on the Pixhawk architecture incurs significant aliasing at almost all frequencies.  Logging the commanded \ac{PWM} values before being sent to I/O CPU solved the aliasing discrepancy and produced very frequency rich models.

The manually piloted acquired data provided viable data source for the models even though it is a very simplistic approach.  There were two separate manual tests run on the same aircraft on the same flight, and the following are the results using this regression technique to model a second-order system:

\begin{equation}
H(s)=\frac{10.39}{s^2+31.26s+504.9}\
\end{equation}

and

\begin{equation}
H(s)=\frac{10.61}{s^2+29.77s+498.7}\
\end{equation}

Converting to standard form as described in Equation~\ref{eq:second_order_model} yields

\begin{equation}
H(s)=\frac{0.0206*22.47^2}{s^2+2*0.69*22.47s+22.47^2}\
\end{equation}

and

\begin{equation}
H(s)=\frac{0.0213*22.33^2}{s^2+2*0.67*22.33s+22.33^2}\
\end{equation}

It is important to note that this system identification technique run on separate sets of data has produced two models with very similar values for $\omega_n$ and $\zeta$.

This produces the average values of

$\omega_n=22.4 rad/s$ \newline
$k = 0.0209$ \newline
$\zeta=0.681$ \newline

With the aliasing removed from the chirped input command signals as previously described, the model is drastically improved and produces the following results:

\begin{equation}
H(s)=\frac{4.409}{s^2+27.11s+430.6}\
\end{equation}

and

\begin{equation}
H(s)=\frac{3.295}{s^2+18.82s+296.5}\
\end{equation}

Converting to standard form as described in Equation~\ref{eq:second_order_model} results in

\begin{equation}
H(s)=\frac{0.0102*20.75^2}{s^2+2*0.65*20.75s+20.75^2}\
\end{equation}

and

\begin{equation}
H(s)=\frac{0.0111*17.21^2}{s^2+2*0.54*17.21s+17.21^2}\
\end{equation}

The comparison of the model to recorded results is seen in Figure~\ref{fig:reverse_chirp_model}.

\begin{figure}[!h]
 \centering
  \includegraphics[width=0.75\textwidth]{reverse_chirp_roll.png}
  \caption{Non-Aliased Reverse Chirp Model Example}
  \label{fig:reverse_chirp_model}
\end{figure}

This produces the average values of

$\omega_n=18.98 rad/s$ \newline
$k = 0.010$ \newline
$\zeta=0.598$ \newline

In the author's experience, these values are reasonable values for this size and weight of airframe based on multiple system identification experiments in \ac{SITL} across multiple airframes as well as actual test flights of three different airframes.  This regression technique has shown potential to create realistic models from actual flight test data.  The data must be properly shaped.  The reverse chirp method has the potential to increase the fidelity of the high-frequency response of the aircraft if the aliasing issue can be resolved on the command input channel.

\chapter{Lyapunov Stability Defintion}
\label{appendix:lyapunov}

Aerospace designs tend to use linear controllers as reference models for their simplicity and well-understood robustness characteristics.  This is despite the fact that the applications of these linear controllers are applied to a non-linear dynamical system such as attitude control with varying dynamic pressure.  Adaptive Controllers are non-linear and may offer performance benefits to non-linear systems as seen in aforementioned aerospace applications.  However, non-linear controller's stability properties need to be evaluated.  The Lyapunov stability criteria offer methods of evaluating these controller's boundedness and robustness behavior.

Aleksandr Lyapunov was a Russian mathematician who's work was published in 1892 \cite{lyapunov1892general} concerning the behavior of non-linear systems close to equilibrium without having to rigorously find the unique solutions to difficult differential equations used to model the system.  His work was largely overlooked until the Cold War when aerospace solutions required a more rigorous approach to analyzing non-linear control robustness.  Modern non-linear control engineers extensively utilize Lyapunov's techniques to design and evaluate non-linear controllers.

\section{Lyapunov Stability Theory}

According to Lyapunov, the stability properties of a system can be classified as stable, asymptotically stable, and exponentially stable.

Given the following autonomous nonlinear system:

 \begin{equation}
\dot{x}(t)=f(x(t)), \qquad x(0)=x_0
\end{equation}

where $f$ has equilibrium at $x_e$ :
 \begin{equation}
f(x_e) = 0
\end{equation}

then the equilibrium is said to be:
\begin{enumerate}
 \item \textbf{Lyapunov Stable} \newline
 for every $\epsilon > 0$ there exists a $\delta > 0$ such that, if \: $\norm{x(0) - x_e} < \delta$, then for every $t \geq 0$ we have  $ \norm{x(t) - x_e} < \epsilon$ 
 \item \textbf{Asymptotically Stable} \newline
 if the system is Lyapunov stable and there exists a $\delta > 0$ such that if \: $\norm{x(0) - x_e}  < \delta$, then $\displaystyle \lim_{t\to \infty} \norm{x(t)-x_e} =0$
 \item \textbf{Exponentially Stable} \newline
 if the system is asymptotically stable and there exists $\alpha > 0, \beta > 0, \delta > 0$ such that if $\norm{x(0)-x_e}<\delta$, then \:$\norm{x(t)-x_e} \leq \alpha \norm{x(0)-x_e} e^{-\beta t}$, for all $t \geq 0$
\end{enumerate}

Being Lyapunov stable infers that if a system is near equilibrium, then it will indefinitely remain near equilibrium.  If the system if found to be asymptotically stable then it eventually will achieve equilibrium as $t\to \infty$ and being exponentially stable implies it reaches equilibrium with a rate of convergence $\beta$.

\subsubsection{Lyapunov's Second Method}
 
Lyapunov's second method is also known as Lyapunov stability criteria.  This method offers a less tenuous method for evaluating mathematically non-ideal systems.  Lyapunov analysis of the linearized system around equilibrium can be cumbersome when the eigenvalues are purely imaginary.  In this case, the solutions can rapidly depart to infinity or approach zero with little perturbation to the eigenvalues.  Lyapunov's second method offers an alternative approach for classifying a system's stability using a concept that is similar to how total energy is defined in classical mechanics.

Conceptually, Lyapunov's second method can be compared to the evaluation of mechanical system similar to the modeling of energy in a vibrating spring mass system.  The energy of the unforced spring mass system will dissipate energy due to friction and or damping.  This trend of energy leaving the system towards some 'attractor' is evidence of the system's stability characteristics and identifies that there will be some stable end state.  Likewise, Lyapunov's second method specifies this with the use of a Lyapunov candidate function $V(x)$ which implicitly characterizes the total energy of the system.  It is important to note that Lyapunov realized that the candidate function could be any positive definite and radially unbounded function.  It is then said to be Lyapunov stable if any candidate function is found and meets the stability criteria.  This means that it is only incumbent upon the engineer to find one candidate function to meet the criteria.  This can be an iterative process of trying various energy like equations.  A common approach is to model the Lyapunov candidate equation as kinetic energy $(\frac{1}{2}u^2)$.  Lyapunov realized that characterizing the energy of a nonlinear system could be almost impossible for some cases, but this approach could prove stability without the rigorous knowledge of the true system's energy.

Lyapunov's second method defines a system as Lyapunov Stable for a system $\dot{x}=f(x)$ having an equilibrium point at $x=0$ where the Lyapunov candidate function $V(x):\mathbb{R}^n \rightarrow \mathbb{R}$ such that:
\begin{itemize}
 \item $V(x)=0$ if and only if $x=0$
 \item $V(x)>0$ if and only if $x\neq0$
 \item $\dot{V}(x)=\frac{d}{dt}V(x)=\sum\limits_{i=1}^{n} \frac{\partial V}{\partial x_i}f_i(x) \leq 0$, for all values of $x\neq 0$  
\end{itemize}

if $\dot{V}(x) < 0$ for $x\neq 0$ then system is asymptotically stable.

Additionally, it is required to demonstrate the condition of radial unboundedness to ensure the system is globally stable \cite{khalil1996noninear}.

\chapter{Projection Operator}
%-------------------------------------------------------------------------------------
%-------------------Setup Matlab Listings syntax---------------------------
%-------------------------------------------------------------------------------------
\definecolor{mygreen}{RGB}{28,172,0} % color values Red, Green, Blue
\definecolor{mylilas}{RGB}{170,55,241}
\lstset{language=Matlab,%
    %basicstyle=\color{red},
    breaklines=true,%
    morekeywords={matlab2tikz},
    keywordstyle=\color{blue},%
    morekeywords=[2]{1}, keywordstyle=[2]{\color{black}},
    identifierstyle=\color{black},%
    stringstyle=\color{mylilas},
    commentstyle=\color{mygreen},%
    showstringspaces=false,%without this there will be a symbol in the places where there is a space
    numbers=left,%
    numberstyle={\tiny \color{black}},% size of the numbers
    numbersep=9pt, % this defines how far the numbers are from the text
    emph=[1]{for,end,break},emphstyle=[1]\color{red}, %some words to emphasise
    %emph=[2]{word1,word2}, emphstyle=[2]{style},    
}
%-------------------------------------------------------------------------------------
%-------------------------------------------------------------------------------------
\section{Euler vs Trapezoidal Method}
%-------------------------------------------------------------------------------------
\begin{lstlisting}[language=matlab]
clear, format long, clc, close all
dt = 1;

t0 = [0:.01:4];

y0 = exp(t0);

%Euler integration

t1 = [0:4];
y1 = exp(0);
for i=1:4
    y1(i+1)=y1(i) + dt*exp(i-1);
end

%Trapezoidal integration
t2 = [0:4];
y2 = exp(0);

for i=1:4
    y2(i+1)=y2(i) + dt*exp(i-1);
    y2(i+1)=y2(i) + (dt/2)*(exp(i-1)+y2(i+1));
end

plot(t0,y0,'k--',t1,y1,'b',t2,y2,'r')
legend('y=e^t','Euler Method','Trapezoidal Method')
\end{lstlisting}

\section{SISO Lyapunov Solution Proof}
%-------------------------------------------------------------------------------------
\begin{lstlisting}[language=matlab]
clear,clc, format compact, close all
%% Find Pb
wn = [5:0.1:10]; %rad/s
for i=1:length(wn)
    b = [wn(i)];
    a = [1,wn(i)];
    [A,B,C,D] = tf2ss(b,a)

    Pb(i) = lyap(A,1);
end
Pb_test = 1./(2*wn);

plot(wn,Pb_test,'o',wn,Pb)

%in this simple 1x1 matrix case Pb is easy to calc by hand in code
% Pb ends up being: Pb=1/2*wn;
\end{lstlisting}

\section{Reverse Linear Chirp}
%-------------------------------------------------------------------------------------
\begin{lstlisting}[language=matlab]
clear, format compact, clc, close all

fs = 1/200;
t = [0:fs:7];
f0= 0.01;%Hz
f1= 10; %Hz
k = (f1-f0)/(t(end));

phi_0 = 0.0;

sample_delay = 0.02;

out = sin(phi_0+2*pi*(f0*(7-(t+sample_delay))+(k/2).*(7-(t+sample_delay)).^2));
%out = 1500 + out*10;

out = out*10;

plot(t,out)
\end{lstlisting}

\section{Projection Operator Example Plots}\label{sec:projection_proof}
%-------------------------------------------------------------------------------------
\begin{lstlisting}[language=matlab]
clear, clc, format compact, close all

%% Plot Projection
epsilon = [0.25,0.28,0.3,0.38];
theta_max = 0.65;
theta_min = 0.25;
center = (theta_max+theta_min)/2;

span = (theta_max-theta_min)*1.4;
theta = [center-(span/2):0.01:center+(span/2)];

for j=1:length(epsilon)
    for i=1:length(theta)          
        %f_theta(i,j) = -(theta(i).^2-theta_max^2)/(epsilon(j)*theta_max^2);
        %f_theta_dot(i,j) = -(2*theta(i))/(epsilon(j)*theta_max^2);
        f_theta(i,j) = -(theta_min-theta(i))*(theta_max-theta(i))./(epsilon(j));
        f_theta_dot(i,j) = (theta_min+theta_max-(2*theta(i)))/(epsilon(j));
    end
end 

figure
for j=1:length(epsilon)   
hold on
plot(theta,f_theta(:,j))
end
line([min(theta),max(theta)],[0,0],'color','blue','linestyle','--')
line([theta_min,theta_min],[min(f_theta(:,1)),max(f_theta(:,1))],'color','red','linestyle','--')
line([theta_max,theta_max],[min(f_theta(:,1)),max(f_theta(:,1))],'color','red','linestyle','--')
legend('\epsilon=0.25','\epsilon=0.28','\epsilon=0.30','\epsilon=0.38')
ylabel('f(\theta)')
xlabel('\theta')
title('Projection Operator')
hold off
\end{lstlisting}


\chapter{Matlab Code}
%-------------------------------------------------------------------------------------
%-------------------Setup Matlab Listings syntax---------------------------
%-------------------------------------------------------------------------------------
\definecolor{mygreen}{RGB}{28,172,0} % color values Red, Green, Blue
\definecolor{mylilas}{RGB}{170,55,241}
\lstset{language=Matlab,%
    %basicstyle=\color{red},
    breaklines=true,%
    morekeywords={matlab2tikz},
    keywordstyle=\color{blue},%
    morekeywords=[2]{1}, keywordstyle=[2]{\color{black}},
    identifierstyle=\color{black},%
    stringstyle=\color{mylilas},
    commentstyle=\color{mygreen},%
    showstringspaces=false,%without this there will be a symbol in the places where there is a space
    numbers=left,%
    numberstyle={\tiny \color{black}},% size of the numbers
    numbersep=9pt, % this defines how far the numbers are from the text
    emph=[1]{for,end,break},emphstyle=[1]\color{red}, %some words to emphasise
    %emph=[2]{word1,word2}, emphstyle=[2]{style},    
}
%-------------------------------------------------------------------------------------
%-------------------------------------------------------------------------------------
\section{Euler vs Trapezoidal Method}
%-------------------------------------------------------------------------------------
\begin{lstlisting}[language=matlab]
clear, format long, clc, close all
dt = 1;

t0 = [0:.01:4];

y0 = exp(t0);

%Euler integration

t1 = [0:4];
y1 = exp(0);
for i=1:4
    y1(i+1)=y1(i) + dt*exp(i-1);
end

%Trapezoidal integration
t2 = [0:4];
y2 = exp(0);

for i=1:4
    y2(i+1)=y2(i) + dt*exp(i-1);
    y2(i+1)=y2(i) + (dt/2)*(exp(i-1)+y2(i+1));
end

plot(t0,y0,'k--',t1,y1,'b',t2,y2,'r')
legend('y=e^t','Euler Method','Trapezoidal Method')
\end{lstlisting}

\section{SISO Lyapunov Solution Proof}
%-------------------------------------------------------------------------------------
\begin{lstlisting}[language=matlab]
clear,clc, format compact, close all
%% Find Pb
wn = [5:0.1:10]; %rad/s
for i=1:length(wn)
    b = [wn(i)];
    a = [1,wn(i)];
    [A,B,C,D] = tf2ss(b,a)

    Pb(i) = lyap(A,1);
end
Pb_test = 1./(2*wn);

plot(wn,Pb_test,'o',wn,Pb)

%in this simple 1x1 matrix case Pb is easy to calc by hand in code
% Pb ends up being: Pb=1/2*wn;
\end{lstlisting}

\section{Reverse Linear Chirp}
%-------------------------------------------------------------------------------------
\begin{lstlisting}[language=matlab]
clear, format compact, clc, close all

fs = 1/200;
t = [0:fs:7];
f0= 0.01;%Hz
f1= 10; %Hz
k = (f1-f0)/(t(end));

phi_0 = 0.0;

sample_delay = 0.02;

out = sin(phi_0+2*pi*(f0*(7-(t+sample_delay))+(k/2).*(7-(t+sample_delay)).^2));
%out = 1500 + out*10;

out = out*10;

plot(t,out)
\end{lstlisting}

\section{Projection Operator Example Plots}\label{sec:projection_proof}
%-------------------------------------------------------------------------------------
\begin{lstlisting}[language=matlab]
clear, clc, format compact, close all

%% Plot Projection
epsilon = [0.25,0.28,0.3,0.38];
theta_max = 0.65;
theta_min = 0.25;
center = (theta_max+theta_min)/2;

span = (theta_max-theta_min)*1.4;
theta = [center-(span/2):0.01:center+(span/2)];

for j=1:length(epsilon)
    for i=1:length(theta)          
        %f_theta(i,j) = -(theta(i).^2-theta_max^2)/(epsilon(j)*theta_max^2);
        %f_theta_dot(i,j) = -(2*theta(i))/(epsilon(j)*theta_max^2);
        f_theta(i,j) = -(theta_min-theta(i))*(theta_max-theta(i))./(epsilon(j));
        f_theta_dot(i,j) = (theta_min+theta_max-(2*theta(i)))/(epsilon(j));
    end
end 

figure
for j=1:length(epsilon)   
hold on
plot(theta,f_theta(:,j))
end
line([min(theta),max(theta)],[0,0],'color','blue','linestyle','--')
line([theta_min,theta_min],[min(f_theta(:,1)),max(f_theta(:,1))],'color','red','linestyle','--')
line([theta_max,theta_max],[min(f_theta(:,1)),max(f_theta(:,1))],'color','red','linestyle','--')
legend('\epsilon=0.25','\epsilon=0.28','\epsilon=0.30','\epsilon=0.38')
ylabel('f(\theta)')
xlabel('\theta')
title('Projection Operator')
hold off
\end{lstlisting}

\section{System Identification}\label{sec:sysid_matlab}
%-------------------------------------------------------------------------------------
\begin{lstlisting}[language=matlab]
clear, clc, format compact, close all

% Import data file for analysis
[filename, pathname] = uigetfile('*.m','Choose first MATLAB file');
run(filename)
%%
% Clean tigger PWMs
for r=1:length(RCIN.data(:,9))    
    if RCIN.data(r,8) < 1700 %channel 6
        RCIN.data(r,8) = 1000;
    end
    if RCIN.data(r,9) < 1700 %channel 7
        RCIN.data(r,9) = 1000;
    end
end

%Trigger Channel Data
roll_trigger_data = RCIN.data(:,9);
pitch_trigger_data = RCIN.data(:,8);
trigger_data_time = RCIN.data(:,2);
%Command Data
roll_command_data = ((RCOU.data(:,3)-1500)+(RCOU.data(:,4)-1500))+1526;
pitch_command_data = ((RCOU.data(:,3)-1500)-(RCOU.data(:,4)-1500))+1460;
command_data_time = RCOU.data(:,2);
%IMU Data
p = IMU2.data(:,3);
q = -IMU2.data(:,4);
IMU_time = IMU2.data(:,2);

% Plot to show sectioned data
plot(RCIN.data(:,2)*10^-6,RCIN.data(:,8)-1500,IMU2.data(:,2)*10^-6,p*(180/pi),...
     RCIN.data(:,2)*10^-6,RCIN.data(:,9)-1500,IMU2.data(:,2)*10^-6,q*(180/pi))
legend('ch6','p','ch7','q')
xlabel('time (seconds)')
ylabel('pwm counts')

[start_index, stop_index] = find_trigger(roll_trigger_data, 1700);
fprintf('Which test would you like to analyze? [Pick a number between 1 and %i]\n',length(start_index));
num_of_test_desired = input('');

%[ input, output, time ] = section_triggered_data( input_data, input_data_time, output_data, output_data_time, trigger_data, trigger_time, num_of_test_desired )
[ input, output, time ] = section_triggered_data( roll_command_data, command_data_time, p, IMU_time, roll_trigger_data, trigger_data_time, num_of_test_desired );

%Center controls for SysID and plotting
input = input-1500; %center stick ouputs 1500

% Theoretical Chirp
fs = 1/50;
t = [0:fs:7];
f0= 0.01;%Hz
f1= 10; %Hz
k = (f1-f0)/(t(end));
phi_0 = 0;

input_theoretical = sin(phi_0+2*pi*(f0*(7-t)+(k/2).*(7-t).^2));
input_theoretical = input_theoretical*75;

figure
plot(time,input,'o', time, output*(180/pi),t,input_theoretical)
legend('roll cmd', 'roll rate')

%% System Identification
loop_rate = mean(diff(IMU_time))*10^-6;
time_clean = [0:loop_rate:(length(time)-1)*loop_rate]';
output_clean = interp1(time,output,time_clean,'pchip');
%input_clean = interp1(time,input,time_clean,'pchip');
input_clean = interp1(t,input_theoretical,time_clean,'pchip');


figure
plot(time, input_clean, time, output_clean*(180/pi))
legend('roll cmd', 'roll rate')

[y,t,x] = estimate_model(input_clean, output_clean, time_clean);

figure
subplot(2,1,1)
plot(time_clean,input_clean)
legend('command')
xlabel('time (seconds)')
ylabel('\delta (pwm)')
title('Roll Analysis')
subplot(2,1,2)
plot(t,y,time_clean,output_clean)
legend('model','measured response')
xlabel('Time (s)')
ylabel('Roll rate (rad/s)')
\end{lstlisting}


%\chapter{Flight Code}
%\label{appendix:system_identification}
\section{System Identification}

\subsection{Data Collection}
The Pixhawk autopilot was used to capture roll and pitch rates $(\dot{p},\dot{q})$ for the test vehicle as well as the pilot's command inputs.  These outputs and inputs were the essential building blocks for creating pitch rate and roll rate models for the test vehicle.  The autopilot is capable of logging data at 50-400 Hz that is represented as a discrete time domain signal.  This data should ultimately be manipulated into the s-domain.  The mathematics for this procedure are well defined, and numerous tools can be used to simplify this process \cite{tfest_matlab}.  

It is crucial to ensure there is sufficient frequency content in the data recorded.  Exciting multiple frequencies in the time domain ensures the regression techniques have an adequate persistence of excitation to resolve polynomial coefficients with higher certainty.  

To ensure sufficient frequency content was obtained from the aircraft, a linear chirp was chosen and implemented into the Pixhawk source code as follows:

\begin{equation}
x(t)=sin\left[\phi_0+2\pi\left(f_0t+\frac{k}{2}t^2\right)\right]
\end{equation}

where
\begin{itemize}
 \item[] $\phi_0$ is the initial phase of the chirp at t=0 (nominally zero degrees)
 \item[] $f_0$ is the initial frequency at t=0
 \item[] $k$ is the chirp rate
 \item[] $t$  is time in seconds
\end{itemize}

An example of this method can be seen in Figure~\ref{fig:chirp}.

\begin{figure}[!h]
 \centering
  \includegraphics[width=0.50\textwidth]{chirp.png}
  \caption{Reverse Linear Chirp Example}
  \label{fig:chirp}
\end{figure}

Specifically for this research, the reverse formulation of the linear chirp was used because it ensured the aircraft wouldn't exceed max roll or pitch limits within the first few cycles of the output.  The reverse chirp was applied in open loop while in \enquote{training mode}  which would constrain the angle of bank or pitch by switching into attitude hold mode upon an attitude exceedance.

\subsection{z-Domain to s-Domain}
The logged input and output data in discrete form requires shaping to convert cleanly into an s-domain representation.  The first step is ensuring that the data is of constant sampling rate.  In other words, the time between samples is uniform from sample to sample.  The data provided from the Pixhawk autopilot does not have a uniform sampling rate.  The sample rate is a user-defined rate (50-400Hz) but has a slight amount of jitter $(\pm 0.1\%)$.  \ac{PCHIP} interpolation was used to interpolate the data into a uniform sampling rate.

After the data is shaped correctly with a uniform sample rate, the discrete data is transformed into a continuous approximation using a \ac{ZOH} technique.  Taking the Laplace transform of the continuous input/output data will convert it into the s-domain, and finally, a non-linear least squares minimization algorithm can be run to find the polynomial coefficients which best fit the data.

The order of the regression (number of polynomials to estimate) is at the discretion of the engineer and their intuition of system's physical representation.  Higher order models will better fit the data, but in most cases, they tend to overfit the data; therefore some tradeoffs should be considered to simplify the model.  Most aircraft models reasonably limit the system to an \ac{LTI} and second-order.  These fundamental aerodynamics models divide the modeling into longitudinal and lateral dynamics.  Each axis of the aircraft is assigned two-second order responses.  Pitch, for example, has a second order response in the pitch damping mode (also known as the short period) and also has a second order response in the transition of kinetic energy to potential energy (also known as the long-period or phugoid).  Both first order and second order models were estimated for comparison sake.

Results were collected from two flight test events.  The first flight test was conducted under manual control without utilizing the chirp.  The pilot attempted to increase frequency of the input signal manually.  The second set of data collected was via the reverse linear chirp method previously described.  

The manually piloted acquired data was expected to have insufficient frequency content in the signal.  However, the manual flight test provided moderate results for modeling the aircraft as seen in Figure~\ref{fig:roll_model} .

\begin{figure}[!h]
 \centering
  \includegraphics[width=0.75\textwidth]{model_response.png}
  \caption{Roll Model Regression with Manual Inputs}
  \label{fig:roll_model}
\end{figure}

The results in Figure~\ref{fig:roll_model} demonstrates the utility of this technique even if data can only be acquired from manual pilot inputs.  It can be seen that the second order model starts to misrepresent the data at higher frequencies.  Figure~\ref{fig:chirp_model} illustrates the data recorded from the reverse chirp experiment run at 50Hz.  The $\delta$ \ac{PWM} input channel data clearly does not represent the software calculated chirp.  The highest frequency designed to be outputted during this experiment was 10Hz.  10 Hz is significantly less than the frequency required to meet Nyquist sampling criterion (25Hz)  for this data sampling rate.  However, there are clear patterns of aliasing in the input signal as recorded by the data flash logger.  

\begin{figure}[!h]
 \centering
  \includegraphics[width=0.75\textwidth]{chirp_response.png}
  \caption{Roll Model Regression with Reverse Linear Chirp}
  \label{fig:chirp_model}
\end{figure}

The chirp response was physically observed on pre-flight, in actual flight, and in the data-flash logged body rate of the aircraft.  However, the aliased input channel was arbitrarily biasing the regression result.  The significant aliasing in the logged input was not due to insufficient sampling rate as previously discussed.  After inspection,  the peculiar aliasing issue was hardware specific to the Pixhawk 1 autopilot in how the main CPU sends servo commands to the auxiliary I/O  CPU.  The most recent version of firmware, at the time of this test, improperly logs the \ac{PWM} through an aliased prone signal path.  The main and I/O CPU both run at 50Hz with some appreciable clock drift.  This generates a noticeable beat frequency and delay when the actual values in registry are saved for \ac{PWM} values are sent back round trip to the main CPU.  The implication of logging the \ac{PWM} values at the very end of the digital transmission line seems valuable in principle because the values being logged are the undeniable values being sent to the actuators.  However, the cost of logging these values in this manner on the Pixhawk architecture incurs significant aliasing at almost all frequencies.  Logging the commanded \ac{PWM} values before being sent to I/O CPU solved the aliasing discrepancy and produced very frequency rich models.

The manually piloted acquired data provided viable data source for the models even though it is a very simplistic approach.  There were two separate manual tests run on the same aircraft on the same flight, and the following are the results using this regression technique to model a second-order system:

\begin{equation}
H(s)=\frac{10.39}{s^2+31.26s+504.9}\
\end{equation}

and

\begin{equation}
H(s)=\frac{10.61}{s^2+29.77s+498.7}\
\end{equation}

Converting to standard form as described in Equation~\ref{eq:second_order_model} yields

\begin{equation}
H(s)=\frac{0.0206*22.47^2}{s^2+2*0.69*22.47s+22.47^2}\
\end{equation}

and

\begin{equation}
H(s)=\frac{0.0213*22.33^2}{s^2+2*0.67*22.33s+22.33^2}\
\end{equation}

It is important to note that this system identification technique run on separate sets of data has produced two models with very similar values for $\omega_n$ and $\zeta$.

This produces the average values of

$\omega_n=22.4 rad/s$ \newline
$k = 0.0209$ \newline
$\zeta=0.681$ \newline

With the aliasing removed from the chirped input command signals as previously described, the model is drastically improved and produces the following results:

\begin{equation}
H(s)=\frac{4.409}{s^2+27.11s+430.6}\
\end{equation}

and

\begin{equation}
H(s)=\frac{3.295}{s^2+18.82s+296.5}\
\end{equation}

Converting to standard form as described in Equation~\ref{eq:second_order_model} results in

\begin{equation}
H(s)=\frac{0.0102*20.75^2}{s^2+2*0.65*20.75s+20.75^2}\
\end{equation}

and

\begin{equation}
H(s)=\frac{0.0111*17.21^2}{s^2+2*0.54*17.21s+17.21^2}\
\end{equation}

The comparison of the model to recorded results is seen in Figure~\ref{fig:reverse_chirp_model}.

\begin{figure}[!h]
 \centering
  \includegraphics[width=0.75\textwidth]{reverse_chirp_roll.png}
  \caption{Non-Aliased Reverse Chirp Model Example}
  \label{fig:reverse_chirp_model}
\end{figure}

This produces the average values of

$\omega_n=18.98 rad/s$ \newline
$k = 0.010$ \newline
$\zeta=0.598$ \newline

In the author's experience, these values are reasonable values for this size and weight of airframe based on multiple system identification experiments in \ac{SITL} across multiple airframes as well as actual test flights of three different airframes.  This regression technique has shown potential to create realistic models from actual flight test data.  The data must be properly shaped.  The reverse chirp method has the potential to increase the fidelity of the high-frequency response of the aircraft if the aliasing issue can be resolved on the command input channel.

% REFERENCES
% List all your BibTeX reference source files (ending in *.bib extension)
%
\NPSbibliography{thesis}


%
% This is the official end of the thesis.
%
\NPSend

% DISTRIBUTION LIST
% The list is automatically properly numbered
% and already populated with the mandatory recipients.
%
\NPSdistribution{Initial Distribution List}
\begin{distributionlist}
\item Defense Technical Information Center\\Ft. Belvoir, Virginia
\item Dudley Knox Library\\Naval Postgraduate School\\Monterey, California
%
%---- Other entries are no longer needed, because of Special Abstract Form
% Marine Corps students are required to show:
%\item Marine Corps Representative\\Naval Postgraduate School\\Monterey, California
%\item Directory, Training and Education, MCCDC, Code C46\\Quantico, Virginia
%\item Marine Corps Tactical System Support Activity (Attn: Operations Officer)\\Camp Pendleton, California
%
% Officer students in the Operations Research Program are also required to show:
%\item Director, Studies and Analysis Division, MCCDC, Code C45\\ Quantico, Virginia
%
% Officer students in the Space Ops/Space Engineering Program or in the Information Warfare/Information Systems and Operations are also required to show:
%\item Head, Information Operations and Space Integration Branch,\\ PLI/PP\&O/HQMC, Washington, DC
\end{distributionlist}


\end{document}

