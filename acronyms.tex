
%% These acronyms are common and do not appear in our list of acronyms
%% We use a different way to define them, so they don't appear in the list:
\newacro{US}[U.S.]{United States}


% Mark the above as used, so they appear in short form by default:
\renewcommand{\NPSacrocommon}{
 \acused{US}
}
\NPSacrocommon{}

%% On the first line below, the longest acronym is placed in the 
%% square brackets.  This is used to size the column of the table 
%% correctly.
%% Other acronyms appear in the list of acronyms, in the order defined.
%%
\begin{acronym}[XXXXXXX] % longest acronym in these square brackets

\acro{DoD}{Department of Defense}
\acro{NPS}{Naval Postgraduate School}
\acro{USN}{\ac{US} Navy}
\acro{SISO}{Single Input Single Output}
\acro{CG}{Center of Gravity}
\acro{LTI}{Linear Time Invariant}
\acro{MAV}{Mirco Aerial Vehcile}
\acro{MEMS}{Microelectromechanical Systems}
\acro{MIMO}{Multiple Input Multiple Output}
\acro{MRAC}{Model Reference Adaptive Control}
\acro{NED}{North East Down}
\acro{PID}{Proportional Integral Differential}
\acro{UAS}{unmanned aerial system}
\acro{IMU}{inertial measurement unit}

\end{acronym}

%% Using documentclass[acronym,...]
%% --------------------------------
%% If you include the 'acronym' package in the documentclass[] options
%% of the NPS template, it will only print items in this acro list that are
%% actually used in your paper.  In this manner, your acronym list will always
%% be up to date.  You can then reuse this acronyms.tex in other documents
%% as you continue to expand your acronyms in use.
%%
%% If you don't include the 'acronym' package explicitly, the package is
%% auto-included by the NPS Thesis Template, but all the acro{} entries
%% will be put into your thesis, whether you actually use them or not.
%%
%% Using the macros:
%% ----------------
%% It will track the usage of acronyms and uses the long form on their
%% first use.  This ensures consistency throughout your document even
%% in different revisions of your document. Your writing is also more portable.
%%
%% To use your acronym in a paragraph use \ac{shortname} or \acplural{shortname}
%% \acp{shortname} for plural shorthand.
%% Ex:  The \ac{FFT2} is performed on the data. 
%% The \acplural{FFT2} are efficient.
%% This command is how LaTeX tracks their usage.
%%
%% Using documentclass[acronym,index,...]
%% --------------------------------------
%% You may want to use this:
%%   \newcommand{\FFT2}{\ac{FFT2}\index{fast Fourier Transform}\xspace}
%% to save some time and typing.  Then you can do:
%% Ex: The \FFT2 is performed on the data.
%% The acronym will be used correctly, and the index will contain the 
%% use of the keyword. 
%%
%% The command's \xspace ensures that the spacing after the word is
%% handled correctly (ie, last word of a sentence gets a period on 
%% the word).


