%-------------------------------------------------------------------------------------
%-------------------Setup Matlab Listings syntax---------------------------
%-------------------------------------------------------------------------------------
\definecolor{mygreen}{RGB}{28,172,0} % color values Red, Green, Blue
\definecolor{mylilas}{RGB}{170,55,241}
\lstset{language=C++,%
    %basicstyle=\color{red},
    breaklines=true,%
    morekeywords={matlab2tikz},
    keywordstyle=\color{blue},%
    morekeywords=[2]{1}, keywordstyle=[2]{\color{black}},
    identifierstyle=\color{black},%
    stringstyle=\color{mylilas},
    commentstyle=\color{mygreen},%
    showstringspaces=false,%without this there will be a symbol in the places where there is a space
    numbers=left,%
    numberstyle={\tiny \color{black}},% size of the numbers
    numbersep=9pt, % this defines how far the numbers are from the text
    emph=[1]{for,end,break},emphstyle=[1]\color{red}, %some words to emphasise
    %emph=[2]{word1,word2}, emphstyle=[2]{style},    
}
%-------------------------------------------------------------------------------------
%-------------------------------------------------------------------------------------
\section{\Lone Adaptive Control Source Code}
%-------------------------------------------------------------------------------------
\begin{lstlisting}[language=c++]
/*
   This program is free software: you can redistribute it and/or modify
   it under the terms of the GNU General Public License as published by
   the Free Software Foundation, either version 3 of the License, or
   (at your option) any later version.
   This program is distributed in the hope that it will be useful,
   but WITHOUT ANY WARRANTY; without even the implied warranty of
   MERCHANTABILITY or FITNESS FOR A PARTICULAR PURPOSE.  See the
   GNU General Public License for more details.
   You should have received a copy of the GNU General Public License
   along with this program.  If not, see <http://www.gnu.org/licenses/>.
 */
/*
  Adaptive controller by Ryan Beall
  See
  http://naira-hovakimyan.mechse.illinois.edu/l1-adaptive-control-tutorials/
  for mathematical basis
 */

#include <AP_HAL/AP_HAL.h>
#include <GCS_MAVLink/GCS.h>
#include "ADAP_Control.h"
#include <stdio.h>

extern const AP_HAL::HAL& hal;

const AP_Param::GroupInfo ADAP_Control::var_info[] = {
    // adaptive control parameters
    AP_GROUPINFO_FLAGS("CH", 1, ADAP_Control, enable_chan, 0, AP_PARAM_FLAG_ENABLE),
    AP_GROUPINFO("AL",       2, ADAP_Control, alpha, 24),
    AP_GROUPINFO("GAMT",     3, ADAP_Control, gamma_theta, 1000),
    AP_GROUPINFO("GAMW",     4, ADAP_Control, gamma_omega, 1000),
    AP_GROUPINFO("GAMS",     5, ADAP_Control, gamma_sigma, 1000),
    AP_GROUPINFO("THU",      6, ADAP_Control, theta_max, 2.0),
    AP_GROUPINFO("THL",      7, ADAP_Control, theta_min, 0.5),
    AP_GROUPINFO("THE",      8, ADAP_Control, theta_epsilon, 5925),
    AP_GROUPINFO("OMU",      9, ADAP_Control, omega_max, 2.0),
    AP_GROUPINFO("OML",     10, ADAP_Control, omega_min, 0.5),
    AP_GROUPINFO("OME",     11, ADAP_Control, omega_epsilon, 5925),
    AP_GROUPINFO("SIGU",    12, ADAP_Control, sigma_max, 0.1),
    AP_GROUPINFO("SIGL",    13, ADAP_Control, sigma_min, -0.1),
    AP_GROUPINFO("SIGE",    14, ADAP_Control, sigma_epsilon, 203),
    AP_GROUPINFO("W0",      15, ADAP_Control, w0, 25),
    AP_GROUPINFO("K",       16, ADAP_Control, k, 0.45),
    AP_GROUPINFO("KG",      17, ADAP_Control, kg, 1.0),
    
    AP_GROUPEND
};

/*
  return true when enabled
 */
bool ADAP_Control::enabled(void) const
{
    // Enables Adaptive Controller instead of PID if rc PWM value is above 1700 milli seconds
    return (enable_chan > 0 && hal.rcin->read(enable_chan-1) >= 1700);
}


/*
  reset to startup values
*/
void ADAP_Control::reset(uint16_t loop_rate_hz)
{
    x_m = x;
    u = 0.0;
    u_lowpass = 0.0;
    u_sp = 0.0;
    theta = 1.0;
    omega = 1.0;
    sigma = 0.0;
    integrator = 0.0;

    float u_cutoff_hz = w0/(2*M_PI); //convert cutoff freq from rad/s to hz
    u_filter.set_cutoff_frequency(loop_rate_hz, u_cutoff_hz);
    u_filter.reset();

    float r_cutoff_hz = (alpha-2)/(2*M_PI); //convert cutoff freq from rad/s to hz
    r_filter.set_cutoff_frequency(loop_rate_hz, r_cutoff_hz);
    r_filter.reset();
}


/*
  trapezoidal integration helper function
 */
float ADAP_Control::trapezoidal_integration(float y0, float y1_dot, float dt, float &y0_dot)
{
    float y1 = y0 + (dt/2)*(y0_dot+y1_dot);
    y0_dot = y1_dot;

    return y1;
}

/*
  adaptive control update. Given a target rate in radians/second and a
  current sensor rate on the same axis in radians/second, return an
  actuator value from -1 to 1
 */
float ADAP_Control::update(uint16_t loop_rate_hz, float target_rate, float sensor_rate, float scaler, float aspeed)
{
    float dt;
    const float u_limit = radians(45);

    x = sensor_rate;
    r = target_rate;
    r = r_filter.apply(r);    

    //reset reference model at initialization
    uint64_t now = AP_HAL::micros64();
    if (last_run_us == 0 || now - last_run_us > 200000UL) {
        reset(loop_rate_hz);
        last_run_us = now;
        return 0;
    }

    dt = (now - last_run_us) * 1.0e-6;
    last_run_us = now;


    // u (controller output to plant)
    eta = theta*x + omega*u_lowpass + sigma;
    u_sp = eta; // u to state predictor

    u = constrain_float(eta-(kg*r),-radians(90)/dt, radians(90)/dt);

    // Check Controller Saturation from previous time step
    bool saturated = ((u < 0 && u_lowpass >= 0.99*u_limit) ||
                      (u > 0 && u_lowpass <= -0.99*u_limit));

    // kD(s) (cascaded second order low pass + simple integrator)
    u_lowpass = u_filter.apply(u);

    if (!saturated) {
    	integrator = trapezoidal_integration(integrator, u_lowpass, dt, out1);
      u_lowpass = constrain_float(-k*integrator,-u_limit,u_limit);
    }

    // State Predictor (first order single pole recursive filter)
    // Reference/Companion Model
    float alpha_filt = exp(-alpha*dt); //alpha in rad/s
    alpha_filt = constrain_float(alpha_filt, 0.0, 1.0);
    float beta_filt = 1-alpha_filt;

    x_m = alpha_filt*x_m + beta_filt*(u_sp);

    x_error = x_m - x;


    // Constrain error to +-300 deg/s
    x_error = constrain_float(x_error,-radians(300), radians(300));

    // Calculate solution to Lyapunov 1x1 matrix
    float Pb = 1/(2*alpha);

    // Projection Operator
    theta_dot = projection_operator(theta,-gamma_theta*x_error*Pb*x,theta_epsilon,theta_max,theta_min);
    omega_dot = projection_operator(omega,-gamma_omega*x_error*Pb*u_lowpass,omega_epsilon,omega_max,omega_min);
    sigma_dot = projection_operator(sigma,-gamma_sigma*x_error*Pb,sigma_epsilon,sigma_max,sigma_min);

    // Parameter Update using Trapezoidal integration
    if (!saturated) {
        theta = trapezoidal_integration(theta, theta_dot, dt, theta1);
        omega = trapezoidal_integration(omega, omega_dot, dt, omega1);
        sigma = trapezoidal_integration(sigma, sigma_dot, dt, sigma1);
    }

    theta = constrain_float(theta, theta_min, theta_max);
    omega = constrain_float(omega, omega_min, omega_max);
    sigma = constrain_float(sigma, sigma_min, sigma_max);

    // Log Data to flash
    DataFlash_Class::instance()->Log_Write(log_msg_name, "TimeUS,Dt,Atheta,Aomega,Asigma,Aeta,Axm,Ax,Ar,Axerr,AuL", "Qffffffffff",
                                           now,
                                           dt,
                                           theta,
                                           omega,
                                           sigma,
                                           eta,
                                           degrees(x_m),
                                           degrees(x),
                                           degrees(r),
                                           degrees(x_error),
                                           degrees(u_lowpass));
 

    return constrain_float(u_lowpass/u_limit, -1, 1);
}

float ADAP_Control::projection_operator(float Theta, float y, float epsilon, float th_max, float th_min) const
{

        // Calculate convex function
        // Nominal un-saturated value is above zero line on a parabolic curve
        // Steepness of curve is set by epsilon
        float f_diff2 = (th_max-th_min)*(th_max-th_min);
        float f_theta = (-4*(th_min - Theta) * (th_max - Theta))/(epsilon*f_diff2);
        float f_theta_dot = (4*(th_min + th_max - (2*Theta)))/(epsilon*f_diff2);

        float projection_out = y;

        if (f_theta <= 0 && f_theta_dot*y < 0)
        {
                projection_out = y*(f_theta+1); //y-(y*(-1*f_theta));
        }

           return projection_out;
}

/*
  send ADAP_TUNING message
*/
void ADAP_Control::adaptive_tuning_send(mavlink_channel_t chan, uint8_t axis)
{
        if (!enabled() || !HAVE_PAYLOAD_SPACE(chan, ADAP_TUNING)) {
        return;
    }
    mavlink_msg_adap_tuning_send(chan, axis,
                                 r,
                                 x,
                                 x_error,
                                 theta,
                                 omega,
                                 sigma,
                                 theta_dot,
                                 omega_dot,
                                 sigma_dot,
                                 x_m,
                                 u_lowpass,
                                 u);
\end{lstlisting}